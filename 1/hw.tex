\mysubsection{Домашняя работа}

\setcounter{example}{0}

\begin{exercise}
	$x, y, z$~---~целые числа, для которых истинно высказывание
\begin{equation}
		\neg(x=y)\wedge((y<x)\rightarrow(2z>x))\wedge((x<y)\rightarrow(x>2z))
		\label{true_eq}
\end{equation}
	Чему равно $x$, если $z=7$, $y=16$?
\end{exercise}

\begin{solution}
	Подставляем из условия значения $z$ и $y$ и преобразуем выражение \mref{true_eq}
	$$
		\neg(x=16)\wedge(\neg(x>16)\vee(x<14))\wedge(\neg(x<16)\vee(x>14)),
	$$
	$$
		(x\neq16)\wedge((x\leqslant16)\vee(x<14))\wedge((x\geqslant16)\vee(x>14)).
	$$
	
	Заметим, что итоговое выражение, как и изначальное, является конъюнкцией
	трех выражений. Тогда оно истинно, если каждое из выражений должно быть
	истинным. Это умозаключение приводит нас к трем условиям:
\begin{enumerate}
	\item $(x\neq16)=1$, если $x\neq16$;
	\item $((x\leqslant16)\vee(x<14))=1$, если $x\leqslant16$;
	\item $((x\geqslant16)\vee(x>14))=1$, если $x>14$.
\end{enumerate}
	
	Пользуясь методом очень пристального взгляда, замечаем, что
	все три условия выше можно переписать так
	$$14<x<16,$$

	\noindent откуда
	$$x=15.$$
\end{solution}

\begin{answer}
	$x=15$
\end{answer}

\begin{exercise}
Постройте таблицу истинности для функции
$$f(x_1,x_2,x_3)=(x_1\vee x_2)\downarrow(x_2\rightarrow x_3)$$
Решение.\\
$x_1$ $x_2$ $x_3$ |$f$\\
0\ \ \ \  0 \ \   0 | 0\\
0\ \ \ \  0 \ \   1 | 0\\
0\ \ \ \  1 \ \   0 | 0\\
0\ \ \ \  1 \ \   1 | 0\\
1\ \ \ \  0 \ \   0 | 0\\
1\ \ \ \  0 \ \   1 | 0\\
1\ \ \ \  1 \ \   0 | 0\\
1\ \ \ \  1 \ \   1 | 0\\
\end{exercise}

\begin{exercise}
Докажите, что 
$$1\oplus x_1\oplus x_2=(x1\rightarrow x_2)\wedge(x_2\rightarrow x_1)$$
Решение.\\
$f_1=1\oplus x_1\oplus x_2$,$f_2=(x1\rightarrow x_2)\wedge(x_2\rightarrow x_1)$\\
\\
$x_1$ $x_2$ |$f_1$|$f_2$\\
0 \ \   0\ \ | 1\ \ | 1 \\
0 \ \   1\ \ | 0\ \ | 0\\
1 \ \   0\ \ | 0\ \ | 0\\
1 \ \   1\ \ | 1\ \ |  1\\
\\
Значит, $f_1=f_2$
\end{exercise}

\begin{exercise}
Докажите формулу
$$\bigvee_{i,j;i\neq j} x_i\oplus x_j=(x_1\vee x_2\vee ...\vee x_n)\wedge(\neg x_1\vee \neg x_2\vee ...\vee \neg x_n)$$
Решение.\\
1)$\bigvee_{i,j;i\neq j} x_i\oplus x_j=1$ $\Rightarrow$ есть как минимум одна пара разных значений($x_i=1$,$x_j=0$). Тогда\\
$(x_1\vee x_2\vee ...\vee x_n)=1$,$\neg x_1\vee \neg x_2\vee ...\vee \neg x_n)=1$ $\Rightarrow$ $(x_1\vee x_2\vee ...\vee x_n)\wedge(\neg x_1\vee \neg x_2\vee ...\vee \neg x_n)=1$\\
2)$\bigvee_{i,j;i\neq j} x_i\oplus x_j=1$ $\Rightarrow$ все $x_i$ и $x_j$ равны 0. Тогда в правой части либо $(x_1\vee x_2\vee ...\vee x_n)=0$, либо $(\neg x_1\vee \neg x_2\vee ...\vee \neg x_n)=0$, а значит и вся правая часть равна 0.\\
\\
ЧТД.
\end{exercise}

\begin{exercise}
Постройте таблицу истинности для f и выразите её через операции $\vee$,$\wedge$,$\neg$, если
$$f=x_1\oplus x_2\oplus x_3\oplus x_1x_2\oplus x_1x_3\oplus x_2x_3\oplus x_1x_2x_3.$$
Решение.\\
\\
$x_1$ $x_2$ $x_3$ |$f$\\
0\ \ \ \  0 \ \   0 | 0\\
0\ \ \ \  0 \ \   1 | 1\\
0\ \ \ \  1 \ \   0 | 1\\
0\ \ \ \  1 \ \   1 | 1\\
1\ \ \ \  0 \ \   0 | 1\\
1\ \ \ \  0 \ \   1 | 1\\
1\ \ \ \  1 \ \   0 | 1\\
1\ \ \ \  1 \ \   1 | 1\\
\\
Перестроние с использованием  $\vee$,$\wedge$,$\neg$: $f_1=x_1\vee x_2\vee x_3$.


\end{exercise}
