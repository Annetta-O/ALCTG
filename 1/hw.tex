\mysubsection{Домашняя работа}

\setcounter{example}{0}

\begin{exercise}
	$x, y, z$~---~целые числа, для которых истинно высказывание
\begin{equation}
		\neg(x=y)\wedge((y<x)\rightarrow(2z>x))\wedge((x<y)\rightarrow(x>2z))
		\label{true_eq}
\end{equation}
	Чему равно $x$, если $z=7$, $y=16$?
\end{exercise}

\begin{solution}
	Подставляем из условия значения $z$ и $y$ и преобразуем выражение \mref{true_eq}
	$$
		\neg(x=16)\wedge(\neg(x>16)\vee(x<14))\wedge(\neg(x<16)\vee(x>14)),
	$$
	$$
		(x\neq16)\wedge((x\leqslant16)\vee(x<14))\wedge((x\geqslant16)\vee(x>14)).
	$$
	
	Заметим, что итоговое выражение, как и изначальное, является конъюнкцией
	трех выражений. Тогда оно истинно, если каждое из выражений должно быть
	истинным. Это умозаключение приводит нас к трем условиям:
\begin{enumerate}
	\item $(x\neq16)=1$, если $x\neq16$;
	\item $((x\leqslant16)\vee(x<14))=1$, если $x\leqslant16$;
	\item $((x\geqslant16)\vee(x>14))=1$, если $x>14$.
\end{enumerate}
	
	Пользуясь методом очень пристального взгляда, замечаем, что
	все три условия выше можно переписать так
	$$14<x<16,$$

	\noindent откуда
	$$x=15.$$
\end{solution}

\begin{answer}
	$x=15$
\end{answer}

\newpage

\begin{exercise}
	Постройте таблицу истинности для функции
\begin{equation}
	f(x_1,x_2,x_3)=(x_1\vee x_2)\downarrow(x_2\rightarrow x_3)
	\label{zero}
\end{equation}
\end{exercise}

\begin{solution}

\begin{paracol}{2}

	Давайте преобразуем выражение \mref{zero}. 
	Для этого представим $x_2\rightarrow x_3$ как 
	$\neg x_2 \vee x_3$. Далее вспомним, что
	$$x\downarrow y = \overline{x\vee y},$$
	
	\noindent откуда получаем, что
	$$f = \neg(x_1\vee x_2 \vee \neg x_2 \vee x_3).$$

	Видно, что под отрицанием стоит дизъюнкция,
	которая на любых наборах будет равна единице,
	поэтому $f$~---~тождественный ноль.

\switchcolumn
		
\begin{center}
 \begin{tabular}{| c c c | c |}
  \hline	
 $x_1$ & $x_2$ & $x_3$ & $f$ \T\B \\
  \hline\hline
  0 & 0 & 0 & 0 \T\B \\ 
  0 & 0 & 1 & 0 \B \\
  0 & 1 & 0 & 0 \B \\
  0 & 1 & 1 & 0 \B \\
  1 & 0 & 0 & 0 \B \\
  1 & 0 & 1 & 0 \B \\
  1 & 1 & 0 & 0 \B \\
  1 & 1 & 1 & 0 \B \\
  \hline
 \end{tabular}
 \vspace{1cm}
 
 \small Таблица истинности функции $f$.
\end{center}

\end{paracol}
\end{solution}

\begin{exercise}
Докажите, что 
\begin{equation}
		1\oplus x_1\oplus x_2=(x_1\rightarrow x_2)\wedge(x_2\rightarrow x_1)
\end{equation}

\end{exercise}

\begin{solution}
 Пусть $f_1=1\oplus x_1\oplus x_2$, $f_2=(x1\rightarrow x_2)\wedge(x_2\rightarrow x_1)$ . Тогда \\
 
\begin{center}
 \begin{tabular}{| c c | c | c |}
  \hline	
 $ x_1$ & $x_2$ & $f_1$ & $f_2$ \\ [0,5 ex]
  \hline\hline
  0 & 0 & 1 & 1\\ [1,5 ex]
  \hline
  0 & 1 & 0 & 0 \\ [1,5 ex]
  \hline
  1 & 0 & 0 & 0 \\ [1,5 ex]
  \hline
  1 & 1 & 1 & 1 \\ [1,5 ex]
  \hline
 \end{tabular}
 \end{center}
Видно, что векторы значений $f_1$ и $f_2$ совпадают, а значит, $f_1=f_2$(т.е. утверждение (1.3) ВЕРНО).
\end{solution}

\begin{exercise}
Докажите формулу
\begin{equation}
		\bigvee_{i,j;i\neq j} x_i\oplus x_j=(x_1\vee x_2\vee ...\vee x_n)\wedge(\neg x_1\vee \neg x_2\vee ...\vee \neg x_n)
\end{equation}

\end{exercise}

\begin{solution}
Рассмотрим 2 случая: \\
\begin{enumerate}
	\item $\bigvee_{i,j;i\neq j} x_i\oplus x_j=1$ $\Rightarrow$ есть как минимум одна пара разных значений($x_i=1$,$x_j=0$). Тогда\\
$(x_1\vee x_2\vee ...\vee x_n)=1$,$\neg x_1\vee \neg x_2\vee ...\vee \neg x_n)=1$ $\Rightarrow$ $(x_1\vee x_2\vee ...\vee x_n)\wedge(\neg x_1\vee \neg x_2\vee ...\vee \neg x_n)=1$;
	\item $\bigvee_{i,j;i\neq j} x_i\oplus x_j=1$ $\Rightarrow$ все $x_i$ и $x_j$ равны 0. Тогда в правой части либо $(x_1\vee x_2\vee ...\vee x_n)=0$, либо $(\neg x_1\vee \neg x_2\vee ...\vee \neg x_n)=0$, а значит и вся правая часть равна 0.
\end{enumerate}
Видно, что векторы значений левой и правой частей равенства совпадают, а значит, формула верна.

\end{solution}



\begin{exercise}
Постройте таблицу истинности для f и выразите её через операции $\vee$,$\wedge$,$\neg$, если
\begin{equation}
		f=x_1\oplus x_2\oplus x_3\oplus x_1x_2\oplus x_1x_3\oplus x_2x_3\oplus 			x_1x_2x_3.
\end{equation}

\end{exercise}

\begin{solution}
Таблица истинности: \\
\begin{center}
 \begin{tabular}{| c c c | c |}
  \hline	
 $ x_1$ & $x_2$ & $x_3$ & $f$ \\ [0,5 ex]
  \hline\hline
  0 & 0 & 0 & 0\\ [1,5 ex]
  \hline
  0 & 0 & 1 & 1 \\ [1,5 ex]
  \hline
  0 & 1 & 0 & 1 \\ [1,5 ex]
  \hline
  0 & 1 & 1 & 1 \\ [1,5 ex]
  \hline
  1 & 0 & 0 & 1 \\ [1,5 ex]
  \hline
  1 & 0 & 1 & 1 \\ [1,5 ex]
  \hline
  1 & 1 & 0 & 1 \\ [1,5 ex]
  \hline
  1 & 1 & 1 & 1 \\ [1,5 ex]
  \hline
 \end{tabular}
 \end{center}
Перестроние с использованием  $\vee$,$\wedge$,$\neg$: \\$f_1=x_1\vee x_2\vee x_3$.
\end{solution}

\begin{answer}
$f_1=x_1\vee x_2\vee x_3$.
\end{answer}