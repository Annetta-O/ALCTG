\begin{center}
\textbf{ \LARGE Мини к/р №1. Алгебра логики: введение. В1}
\end{center}
\\
\\
\textbf{Ex. 1.} Постройте таблицу истинности для функции
$ f = (x_1 \oplus x_2 ) \rightarrow x_3 $
\\
\\
Требуемая таблица представлена ниже:
\\
\begin{center}
\resizebox{10cm}{!}{
\begin{tabular}{ | c | c | c | c | c | }
\hline
$ x_1 $ & $ x_2 $ & $ x_3 $ & $ ( x_1 \oplus x_2 ) $ & f \\ \hline
0 & 0 & 0 & 0 & 1 \\
0 & 0 & 1 & 0 & 1 \\
0 & 1 & 0 & 1 & 0 \\
0 & 1 & 1 & 1 & 1 \\
1 & 0 & 0 & 1 & 0 \\
1 & 0 & 1 & 1 & 1 \\
1 & 1 & 0 & 0 & 1 \\
1 & 1 & 1 & 0 & 1 \\
\hline
\end{tabular}
}
\end{center}
\\
\\
\\
\textbf{Ex. 2.} Докажите формулу разложения:
\\
\\
$ f(x_1, ... , x_n) = (x_1 \vee f(0, x_2, ... , x_n)) \wedge (\neg x_1 \vee f(1, x_2, ... , x_n)) $
\\
\\
Пусть переменная x_1 -- фиктивная. Тогда:
\\
$ f(0, ... , x_n) = f(1, ... , x_n) $
\\
Если $ f(x) = 1 $ при фиксированных $ x_2, ... , x_n $
\\
$ 1 = (x_1 \vee 1) \wedge (\neg x_1 \vee 1) $
\\
$ 1 = 1 \wedge 1 = 1 $
\\
Если $ f(x) = 0 $ при фиксированных $ x_2, ... , x_n $
\\
$ 0 = (x_1 \vee 0) \wedge (\neg x_1 \vee 0) $
\\
$ 0 = x_1 \wedge \neg x_1 = 0 $
\\
Пусть переменная x_1 -- существенная. Тогда:
\\
$ f(0, ... , x_n) \neq f(1, ... , x_n) $
\\
При $ f(0, ... , x_n) = 1,\; f(1, ... , x_n) = 0 $
\\
$ f(x_1, ... , x_n) = (0 \vee x_1) \wedge (1 \vee x_1) = x_1 \wedge 1 = x_1 $
\\
Обратное допущение доказывается аналогично.
\\
\\
\textbf{Ex. 3.} Выразите конъюнкцию и дизъюнкцию, используя \underline{только} штрих Шеффера.
\\
\\
а) Сравним векторы значений функций $ f_1 = ( x \wedge y) $ и $ f_2 = (x \; | \; y) $ :
\\
$ f_1 = 0001 $, $ f_2 = 1110$
\\
f_2 истинна, если оба члена функции не истинны одновременно. Тогда, чтобы добиться вектора значений f_1, достаточно использовать структуру $ f_2( f_2, f_2) $ или $ (x  \; | \; y) \; | \; (x \; | \; y) $
\\
\\
\resizebox{6cm}{!}{
\begin{tabular}{ | c | c | c | c | c | }
\hline
$ x $ & $ y $ & $ (x \; | \; y) $ & $ (x \; | \; y) $ & $ f_2(f_2, f_2)$ \\ \hline
0 & 0 & 1 & 1 & 0 \\
0 & 1 & 1 & 1 & 0 \\
1 & 0 & 1 & 1 & 0 \\
1 & 1 & 0 & 0 & 1 \\
\\
\hline
\end{tabular}
}
\\
\\
Данная таблица значений совпадает с конъюнкцией.
\\
\\
б) Пусть $ f_3 = x \vee y $
\\
f_3 = 0111
\\
x = 0101
y = 0011
\\
При $ f_2 (x, x) $ получаем вектор значений 1010 (по аналогии с инверсией, т.к. штрих Шеффера при 2 истинах возвращает ложь.
\\
При $ f_2 (y, y) $ вектор значений -- 1100
\\
Очевидно, для того, чтобы в результате получился вектор значений 0111, достаточно использовать штрих Шеффера на $ f_2 (x, x) $ и $ f_2 (y, y) $ (при нем первые значения соответствующих функций равны истине и обращаются в ложь, а остальные дают истину)
\\
Т.е. $(x \; | \; x) \; | \; (y \; | \; y)$ и есть искомая функция.
\\
\\
\textbf{Ex. 4.} 