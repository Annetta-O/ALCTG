\mysubsection{Домашняя работа}

\setcounter{example}{0}

\begin{exercise}
Постройте СДНФ и СКНФ для функции $(xz\oplus\overline{y})\equiv(x\to y)$.
\end{exercise}

\begin{solution}
\end{solution}

\begin{exercise}
Постройте замыкание базиса $\{\neg,\oplus\}$.
\end{exercise}

\begin{solution}
\end{solution}

\begin{exercise}
Укажите существенные и несущественные (фиктивные) переменные функции 
	$f(x_1,x_2,x_3)=00111100$ и разложите ее в ДНФ и КНФ.
\end{exercise}

\begin{solution}
\end{solution}

\begin{exercise}
Докажите или опровергните полноту системы функций $\{\oplus,\to\}$.
\end{exercise}

\begin{solution}
\end{solution}

\begin{exercise}
Пусть $f(x_1,\ldots,x_n)$~---~несамодвойственная функция. Докажите, что
	константы $0, 1$ вычисляются в базисе $\{\neg, f\}$.
\end{exercise}

\begin{solution}
\end{solution}

\begin{exercise}
Запишите в виде КНФ функцию от $n$ переменных, принимающую значение
	$0$ лишь на $\vec{0}$ и на $\vec{1}$. Покажите, что эта функция равна
	дизъюнкции всевозможных скобок $(x_i\oplus x_j)$, где $i\neq j$. 
\end{exercise}

\begin{solution}
\end{solution}

\begin{exercise}
Функцию алгебры логики называют \textit{симметрической,} если она не 
	меняет своего значения при любой перестановке значений переменных местами.
	Покажите, что функция $\overline{xy}\vee\overline{yz}\vee\overline{zx}$~---~симметрическая.
	Найдите число симметрических функций от $n$ переменных.
\end{exercise}

\begin{solution}
\end{solution}

\begin{exercise}
Докажите, что любая неконстантная симметрическая функция существенно зависит
от всех своих переменных.
\end{exercise}

\begin{solution}
\end{solution}

\begin{exercise}
Докажите, что если система $\{f_1,\ldots,f_n\}$ полна, то и система
двойственных функций $\{f_1^*, \ldots, f_n^*\}$ также полна.
\end{exercise}

\begin{solution}
\end{solution}

