\mysubsection{Домашняя работа}

\setcounter{example}{0}

\begin{exercise}
Постройте СДНФ и СКНФ для функции $(xz\oplus\overline{y})\equiv(x\to y)$.
\end{exercise}

\begin{solution}
Построим таблицу истинности:\\



\begin{center}
 \begin{tabular}{| c c c | c | c | c | c | c |}
  \hline	
 $x$ & $y$ & $z$ & $x\wedge y$ & $\bar{z}$ & $xz\oplus \bar{z}$ & $x\rightarrow y$ & $f$\T\B \\
  \hline\hline
  0 & 0 & 0 & 0 & 1 & 1 & 1 & 1 \T\B \\ 
  0 & 0 & 1 & 0 & 1 & 1 & 1 & 1\B \\
  0 & 1 & 0 & 0 & 0 & 0 & 1 & 0\B \\
  0 & 1 & 1 & 0 & 1 & 1 & 0 & 0 \B \\
  1 & 0 & 0 & 0 & 0 & 0 & 1 & 0\B \\
  1 & 0 & 1 & 1 & 1 & 0 & 0 & 1 \B \\
  1 & 1 & 0 & 0 & 0 & 0 & 1 & 0 \B \\
  1 & 1 & 1 & 1 & 0 & 1 & 1 & 1 \B \\
  \hline
 \end{tabular} 
 \vspace{0,2 cm}
 
 \
 \small Таблица истинности функции.
 
 \end{center}
 \begin{enumerate}
\item $f=x^0y^0z^0\vee x^0y^0z^1\vee x^1y^0z^1\vee x^1y^1z^1$
\item $f=\bar{x}\bar{y}\bar{z}\vee \bar{x}\bar{y}z\vee x\bar{y}z\vee xyz$ ~---~СДНФ
\item$f=(x^1\vee y^0\vee z^1)\wedge(x^0\vee y^1\vee z^1 )\wedge(x^1\vee y^0\vee z^0)\wedge(x^0\vee y^0\vee z^1)$
\item$f=(x\vee \bar{y}\vee z)\wedge(\bar{x}\vee y\vee z)\wedge(x\vee \bar{y}\vee z)\wedge(\bar{x}\vee \bar{y}\vee z)$~---~СКНФ
 \end{enumerate}
 
\end{solution}

\begin{exercise}
Постройте замыкание базиса $\{\neg,\oplus\}$.
\end{exercise}

\begin{solution}
Построив таблицы истинности $f_1=\neg x$,$f_2=x\oplus 1$, получим, что их векторы значений совпадают, значит, $\neg x=x\oplus 1$.Тогда
$$[{\neg , \oplus}]=[{1,\oplus}]\subseteq L$$.
\end{solution}

\begin{exercise}
Укажите существенные и несущественные (фиктивные) переменные функции 
	$f(x_1,x_2,x_3)=00111100$ и разложите ее в ДНФ и КНФ.
\end{exercise}

\begin{solution} Построим таблицу истинности и напишем сначала СДНФ и СКНФ данной функции $f$:

\begin{center}
 \begin{tabular}{| c c c | c |}
  \hline	
 $x_1$ & $x_2$ & $x_3$ & $f$ \T\B \\
  \hline\hline
  0 & 0 & 0 & 0 \T\B \\ 
  0 & 0 & 1 & 0 \B \\
  0 & 1 & 0 & 1 \B \\
  0 & 1 & 1 & 1 \B \\
  1 & 0 & 0 & 1 \B \\
  1 & 0 & 1 & 1 \B \\
  1 & 1 & 0 & 0 \B \\
  1 & 1 & 1 & 0 \B \\
  \hline
 \end{tabular}
 \vspace{0,4 cm}
 
 \
 \small Таблица истинности функции $f$.
\end{center}
\begin{enumerate}
\item$f=x_1^0x_2^1x_3^0\vee x_1^0x_2^1x_3^1\vee x_1^1x_2^0x_3^0\vee x_1^0x_2^0x_3^1$,\\
$f=\bar{x_1}x_2\bar{x_3}\vee \bar{x_1}x_2x_3\vee x_1\bar{x_2}\bar{x_3}\vee x_1\bar{x_2}x_3$~---~CДНФ.
\item$f=(x_1^1\vee x_2^1\vee x_3^1)\wedge(x_1^1\vee x_2^1\vee x_3^0)\wedge(x_1^0\vee x_2^0\vee x_3^1)\wedge(x_1^0\vee x_2^0\vee x_3^0)$,\\
$f=(x_1\vee x_2\vee x_3)\wedge(x_1\vee x_2\vee \bar{x_3})\wedge(\bar{x_1}\vee \bar{x_2}\vee x_3)\wedge(\bar{x_1}\vee \bar{x_2}\vee \bar{x_3})$~---~СКНФ.\\

\end{enumerate}
Из таблицы истинности видно, что если брать одинаковые значения  $x_1$ и $x_2$, то значени функции при изменении значений $x_3$ меняться не будет. Значит, $x_3$ - фиктивная переменная. Про $x_1$, $x_2$ такого сказать нельзя, поэтому они являются существенными  переменными. Тогда:
\begin{enumerate}
\item КНФ: $f=(x_1\vee x_2)\wedge(x_1\vee x_2)\wedge(\bar{x_1}\vee \bar{x_2})\wedge(\bar{x_1}\vee \bar{x_2}\vee)$
\item ДНФ: $f=\bar{x_1}x_2\vee \bar{x_1}x_2\vee x_1\bar{x_2}\bar\vee x_1\bar{x_2}$
\end{enumerate}


\end{solution}

\begin{exercise}
Докажите или опровергните полноту системы функций $\{\oplus,\to\}$.
\end{exercise}

\begin{solution}
Выразим функции $\oplus$ и $\to$ через $\vee$,$\wedge$,$\neg$:
\begin{enumerate}
\item $(x\to y)\equiv(\neg x\vee y)$,
\item $(x\oplus y)\equiv(x\bar{y}\vee \bar{x}y)$
\end{enumerate}
Значит, система функций $\{\oplus,\to\}$ полна.
\end{solution}

\begin{exercise}
Пусть $f(x_1,\ldots,x_n)$~---~несамодвойственная функция. Докажите, что
	константы $0, 1$ вычисляются в базисе $\{\neg, f\}$.
\end{exercise}

\begin{solution}
Так как $f$~---~несамодвойственная функция, то существует  последовательность значений /{$a_1,...,a_n$/} из нулей и единиц, такая что $f(a_1,...,a_n)=f(\bar{a_1},...,\bar{a_n})=c$, где $c=0$ или $c=1$;\\
Подставим для любых i в качестве аргументов в f:\\
$x_i=x$, если $a_i=1$ и $x_i=\bar{x}$, если $a_i=0$;\\
$g(x)=f(x_1,...,x_n)$:
\begin{enumerate}
\item $g(1)=f(a_1,...,a_n)$
\item $g(0)=f(\bar{a_1},...,\bar{x_n})$
\end{enumerate} 
$g(x)=c$, а $\bar{g(x)}=\bar{c}$.
\end{solution}

\begin{exercise}
Запишите в виде КНФ функцию от $n$ переменных, принимающую значение
	$0$ лишь на $\vec{0}$ и на $\vec{1}$. Покажите, что эта функция равна
	дизъюнкции всевозможных скобок $(x_i\oplus x_j)$, где $i\neq j$. 
\end{exercise}

\begin{solution}
КНФ:\\
$(x_1 \vee ...\vee x_n)\wedge(x_1 \vee ...\vee xn_)$\\
Покажем, что функция равна дизъюнкции всевозможных скобок $(x_i\oplus  x_j )$ для всех $i\neq j$
\begin{enumerate}
\item если $x_i=x_j$ :\\
$(0\oplus 0) \vee  ... \vee (0\oplus0) = 0$;\\
$(1 \oplus 1) \vee ... \vee (1\oplus 1) = 0$;
\item найдется скобка, где $x_i\neq x_j$ :\\
$(0 \oplus 0) \vee ... \vee (0 \oplus 1) \vee ... \vee (1 \oplus 1) = 1$;
\end{enumerate}
\end{solution}

\begin{exercise}
Функцию алгебры логики называют \textit{симметрической,} если она не 
	меняет своего значения при любой перестановке значений переменных местами.
	Покажите, что функция $\overline{xy}\vee\overline{yz}\vee\overline{zx}$~---~симметрическая.
	Найдите число симметрических функций от $n$ переменных.
\end{exercise}

\begin{solution}
Построим таблицу истинности для функции $f$:
\begin{center}
 \begin{tabular}{| c c c | c | c | c | c |}
  \hline	
 $x$ & $y$ & $z$ & $\bar{xy}$ & $\bar{xz}$ & $\bar{yz}$ & $f$ \T\B \\
  \hline\hline
  0 & 0 & 0 & 1 & 1 & 1 & 1 \T\B \\ 
  0 & 0 & 1 & 1 & 1 & 1 & 1 \B \\
  0 & 1 & 0 & 1 & 1 & 1 & 1 \B \\
  0 & 1 & 1 & 1 & 1 & 1 & 1 \B \\
  1 & 0 & 0 & 1 & 1 & 0 & 1 \B \\
  1 & 0 & 1 & 1 & 0 & 1 & 1\B \\
  1 & 1 & 0 & 0 & 1 & 1 & 1 \B \\
  1 & 1 & 1 & 0 & 0 & 0 & 0 \B \\
  \hline
 \end{tabular}
 \vspace{0,4 cm}
 
 \
 \small Таблица истинности функции $f$.
\end{center}

В каждом из наборов, где количество единиц меняется от 0 до 3 в наборе, функция принимает одинаковое значение. То есть на наборах $(0,0,1)$, $(0,1,0)$,$(1,0,0)$ функция не меняет значение. Аналогично для наборов, где единицы 2 единицы из трех. Поменяв местами переменные, вектор значений не изменился, значит, функция является симметрической.\\

Симметрических функций от n переменных существует $2^{n+1}$ штук, потому что существует 2 варианта значения функции, не зависящих от расположения переменных на одном наборе. Таких наборов $n+1$ штук, а значит, симметрических функций от n переменных $2^{n+1}$.
\end{solution}

\begin{exercise}
Докажите, что любая неконстантная симметрическая функция существенно зависит
от всех своих переменных.
\end{exercise}

\begin{solution}
Предположим обратное. Значит неконстантная симметрическая функция несущественно зависит от переменных, то есть
$$f(x_1,..., x_{i-1}, 1, ... x_n) = f(x_1,..., x_{i-1}, 0, ... x_n)$$, а так как эта функция симметрическая, то от перестановки переменных значение ее меняться не будет: $f(\overleftarrow{x})=0 \vee  f(\overleftarrow{x})=1$, что противоречит высказыванию о том, что функция неконстантная.  Значит, неконстантная симметрическая функция существенно зависит от переменных.
\end{solution}

\begin{exercise}
Докажите, что если система $\{f_1,\ldots,f_n\}$ полна, то и система
двойственных функций $\{f_1^*, \ldots, f_n^*\}$ также полна.
\end{exercise}

\begin{solution}
Донт ноу!!!!!!!!!!!!!!!!!!!!!
\end{solution}

