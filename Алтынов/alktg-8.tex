\documentclass[a4paper,12pt]{article}

\usepackage[T2A]{fontenc}
\usepackage[utf8]{inputenc}
\usepackage[english.russian]{babel}
\usepackage{graphicx}
\graphicspath{{pictures/}}
\DeclareGraphicsExtensions{.pdf,.png,.jpg}

\usepackage{amsmath,amsfonts,amssymb,amsthm,mathtools}

\author{Altunov A.}
\title{АЛКТГ - 8 задание
\\
Комбинаторика - 2}
\date{\today}

\begin{document}

\maketitle
\newpage
\textbf{Ex.1.} Робот ходит по координатной плоскости. На каждом шаге он
может увеличить одну координату на 1 или обе координаты на 2.
Сколько есть способов переместить Робота из точки (0, 0) в точку
(4, 5)?
\\

Напишем для каждой оси координат робота кол-во действий для достижения нужной точки:
\\
x : $ a_1 + 2a_2 = 4 $
\\
y : $ b_1 + 2a_2 = 5 $
\\
где $a_1$ -- кол-во ходов типа (x, y) --> (x+1, y), 
\\
$a_2$ -- типа (x, y) --> (x+2), (y+2), 
\\
$b_1$ -- типа (x, y) --> (x, y+1)
\\
Решая система в неотрицательных числах, получаем ограничения и зависимость $a_1$, $b_1$:
\\
\begin{equation*}
b_1 - a_1 = 1
\end{equation*}
\begin{equation*}
a_1 \leq 4 
\end{equation*}
\begin{equation*}
b_1 \leq 5 
\end{equation*}
\\
С учетом этих уточнений найдем все возможные варианты значений $ a_1, b_1, a_2 $. Их оказалось три:
\\
1) $ a_1 = 0, a_2 = 2, b_1 = 1 $
\\
2) $ a_1 = 2, a_2 = 1, b_1 = 3 $
\\
3) $ a_1 = 4, a_2 = 0, b_1 = 5 $
\\
Для каждого случая общее число вариантов ходов -- перестановки с повторениями:
\\
(1) $ P_1 = \dfrac{3!}{0!1!2!} = 3 $; 
(2) $ P_2 = \dfrac{6!}{1!2!3!} = 60 $;
(3) $ P_2 = \dfrac{9!}{0!4!5!} = 126 $;
\\
$ P = P_1 + P_2 + P_3 = 3 + 60 + 126 = 189 $
\\

\textbf{Ответ. 189 вариантов}
\\

\textbf{Ex.2.} В магазине продается 10 видов пирожных. Сколькими спосо-
бами можно купить 100 пирожных (порядок покупки не важен)?
\\

Искомое кол-во вариантов -- сочетания с повторениями (в роли перегородок различие между видами пирожных)
\\
По формуле находим: $ \overline C_n^k = C_{n+k-1}^{k-1} = C_{109}^{9}$
\\
\textbf{Ответ. $ C_{109}^{9} $ вариантов}
\\

\textbf{Ex.3.} Сколько различных слов (не обязательно осмысленных) можно
получить, переставляя буквы в словах
\\

a) <<ЛИНИЯ>>
\\
Разберем повторения букв в слове: Л -- 1, И -- 2, Н -- 1, Я -- 1
\\
По формуле перестановок с повторениями получаем: $ P = \dfrac{5!}{1!2!1!1!} = 60 $

b) <<ОБОРОНОСПОСОБНОСТЬ>>
\\
Аналогично определяем повторения:
\\
 О -- 7, Б -- 2, Р -- 1, Н -- 2, С -- 3, П -- 1, Т -- 1, Ь -- 1
\\
$ P = \dfrac{18!}{7!2!2!3!} $
\\
\textbf{Ответ. a) 60 вариантов. b) $ \dfrac{18!}{120960} $ вариантов.}
\\

\textbf{Ex.4.} а) Как связаны между собой $ C_n^k $ и $ C_n^{k+1} $?
\\

Т.к $ C_n^k = \dfrac{n!}{k!(n-k)! } $, $ C_n^{k+1} = \dfrac{n!}{(k+1)!(n-k-1)!} $
\\
\\
\\
$ \dfrac{C_n^{k+1}}{C_n^k} = \dfrac{n!k!(n-k)!}{n!(k+1)!(n-k-1)!} = \dfrac{(n-k)}{(k+1)} $
\\

b) Какое из слагаемых в разложении $(1 + 2)^n$ по формуле Бинома Ньютона будет наибольшим?
\\

Рассмотрим отношение очередного слагаемого в разложении Бинома к предыдущему и исследуем, когда оно больше единицы:
\\
$ \dfrac{x_{k+1}}{x_k} = \dfrac{C_n^{k+1}*2^{k+1}}{C_n^k*2^k} > 1 $
\\
С учетом полученного из предыдущего пункта соотношения:
\\
$ 2* \dfrac{n-k}{k+1} > 1 $ $\Rightarrow$ 
$ k < \dfrac{2n-1}{3} $
\\
Значит следующее максимальным целым к слагаемое будет максимальным, все последующие монотонно убывают.
\\
Требуемый номер: $ k = \lfloor \dfrac{2n-1}{3} \rfloor + 1 $
\\
Искомое слагаемое: $ x_k = 2^{\lfloor \dfrac{2n-1}{3} \rfloor + 1} * C_n^{\lfloor \dfrac{2n-1}{3} \rfloor + 1} $
\\
\\
\textbf{Ответ. $ x_{max} = 2^{\lfloor \dfrac{2n-1}{3} \rfloor + 1} * C_n^{\lfloor \dfrac{2n-1}{3} \rfloor + 1} $}
\\

\textbf{Ex.5.} Дайте комбинаторное доказательство тождеств:
\\

a) $ C_n^m * C_m^k = C_n^k * C_{n-k}^{m-k} $
\\
Задача: Имеется $n$ мест, в любое из которых можно положить 1 шар красного или синего цвета.  Всего m шаров. Из них k -- красные, $m-k$ -- синие. Найти кол-во вариантов расстановки шаров.
\\
Мы можем выбрать m мест под шары из n возможных ($ C_n^m $ вариантов). В полученных местах распределить k красных шаров, синие тогда распределятся однозначно ($ C_m^k $ вариантов).
\\
Всего вариантов выходит: $ C_n^m * C_m^k $.
\\
С другой стороны, выберем m мест из n для расстановки красных шаров ($ C_n^k $ вариантов). Из оставшихся $ n - k $ мест выберем $ m-k $ для расстановки синих шаров ($ C_{n-k}^{m-k} $).
\\
Всего вариантов: $ C_n^k * C_{n-k}^{m-k} $
\\
Таким образом, $ C_n^m * C_m^k = C_n^k * C_{n-k}^{m-k} $, ч.т.д.
\\

\textbf{Ex.6.} Сформулируем задачу. Пусть мы имеем кузнечика, прыгающего по координатной прямой. Он может прыгать на следуюшюю кординату и через одну. Найти кол-во вариантов добраться в точку с координатой n.
\\
Кол-во вариантов добраться на координату n: $ F_n = F_{n-1} + F_{n-2} $ т.к. в координату n кузнечик может попасть из n-1 или n-2. Это числа Фибоначчи.
\\
С другой стороны, пусть k -- кол-во прыжков через одну координату или число пропущенных координат. Тогда кол-во вариантов добраться до $n+1$ -ой координаты находится по формуле сочетаний с повторениями: $ C_{n+1-k}^k $. Т.к. кол-во пропущенных координат не может превышать $ \dfrac{n+1}{2} $, искомое кол-во вариантов -- это сумма сочетаний с каждым из k от 0 до $ \dfrac{n+1}{2} $. Таким образом, искомое равенство кол-в вариантов через сочетания и числа Фибоначчи достигаетя, ч.т.д.
\\

\textbf{Ex. 7.} Сколько существует последовательностей из нулей и единиц
длины 16, в которых никакие три единицы не стоят рядом?
\\

Найдем рекуррентное соотношение кол-ва вариантов строки от ее длины.
\\
Пусть мы имеем строку из n символов, которая подчиняется правилу из условия и имеет $P_n$ корректных вариантов формирования. Добавим к этой строке один символ a. Так как у добавочного символа могут быть два значения -- 0 и 1, кол-во вариантов новой строки умножается на 2: $ P_n * 2 $. Но среди этих вариантов могут быть некорректные: когда в изначальное строке последние два символа -- единицы, и добавочный символ -- единица. Эти варианты надо вычеркнуть. Заметим, что в некорректных вариантах зафиксированы три последние символа изначальной строки: последние два символа -- единицы, а перед ними -- ноль (это гарантирует корректность изначальной строки). Тогда некорректными являются все варианты строки n с зафиксированными 3 символами, т.е. $ P_{пл} = P_{n-3} $
\\
Общее кол-во корректных вариантов: $ P_n * 2 - P_{n-3} $
\\
При $P_0 = 1$, $P_1 = 2$, $P_2 = 4$, $P_3 = 7$ рекуррентно находя варианты, получим ответ.
\\
\textbf{Ответ. 19513 вариантов}
\\

\end{document}
