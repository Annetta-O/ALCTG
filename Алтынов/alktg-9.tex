\documentclass[a4paper,12pt]{article}

\usepackage[T2A]{fontenc}
\usepackage[utf8]{inputenc}
\usepackage[english.russian]{babel}
\usepackage{graphicx}
\graphicspath{{pictures/}}
\DeclareGraphicsExtensions{.pdf,.png,.jpg}
\usepackage[left=2cm,right=2cm,
    top=2cm,bottom=2cm,bindingoffset=0cm]{geometry}
\usepackage{amsmath,amsfonts,amssymb,amsthm,mathtools}

\author{Altunov A.}
\title{АЛКТГ - 9 задание
\\
Комбинаторика - 3}
\date{\today}

\begin{document}

\maketitle
\newpage
\textbf{Ex.1.} В корзине лежат 60 шаров различных цветов. Если взять любые 10, то обязательно среди них найдется три одноцветных. Обязательно ли среди всех 60 шаров найдется: а) 15 одноцветных; б) 16 одноцветных?
\\

a) Допустим противное. Пусть каждый цвет используется не более 14 раз. Найдем максимальное кол-во цветов, при котором выполняется условие. Для трех цветов чтобы встретилось 3 цвета в выборке, достаточно $ 2 * 3 + 1 = 7 $ шаров в выборке (по принципу Дирихле). Для 4 цветов достаточно 9 шаров. Для пяти цветов -- 11, слишком много. Заметим, что при использовании 4 цветов один шар в выборке не влияет на результат, его можно заменить шаром нового цвета. Таким образом мы добавляем 5 цвет в виде одного шара (2 шара добавить уже нельзя, иначе может встретится выборка из 10 шаров, каждые два из которых одинакового цвета. На оставшиеся 4 цвета есть $ 60 - 1 = 59 $ шаров. Оставшиеся 4 цвета по условию имеют по условию не более 14 шаров на каждый. Тогда максимум шаров 4 цветов $ 4 * 14 = 56 $. Но их должно быть 59. Тогда по принципу Дирихле хотя бы один цвет имеет 15 шаров. Противоречие. Следовательно, при даже с максимально возможным кол-вом цветов существует цвет, который имеют 15 шаров. Значит Для меньшего кол-ва цветов данное условие выполнится автоматически.
\\

б) В предыдущем пункте мы доказали, что при максимальном кол-ве цветов хотя бы один цвет встречается 15 раз. Пусть кол-во шаров каждого цвета не превосходит 15. Докажим, что такое возможно. Из предыдущего пункта возьмем 5 цветов, обозначим их как: a,b,c,d,e. Кол-во шаров a -- 1, b -- 15 (доказано в пред. пункте). На c,d,e остается 44 шара. При этом $ a,b,c \leq 15 $. Пусть c -- 15, d -- 15, e -- 14. Гарантируется, что даже при выборке из всех 5 цветов у нас выполняется условие. При меньшем кол-ве цветов в выборке выполнение очевидно. Таким образом, мы привели пример раскраски шаров, при котором ни один цвет не имеет более 15 шаров, ч.т.д. 
\\
\textbf{Ответ. а) Обязательно. б) Необязательно.}
\\

\textbf{Ex.2.} В группе студентов есть один, который знает C++, Java, Python, Haskell. Каждые три из этих языков знают два студента. Каждые два -- 6 студентов. Каждый из этих языков знают по 15 студентов. Каково наименьшее количество студентов в такой группе?
\\

Минимальное количество студентов образуется при допущении, что нет ни одного студента, не знающего ни одного языка. Используем этот факт. Пусть множество $A_1$ -- студенты, знающие С++, $A_2$ -- Java, $A_3$ -- Python, $A_4$ -- Haskell. Тогда $N(\overline A_1, \overline A_2, \overline A_3, \overline A_4  ) = 0$ (по нашему допущению). С другой стороны, по формуле включений-исключений:
\\
 $N(\overline A_1, \overline A_2, \overline A_3, \overline A_4  ) = N_0 - A_1 - A_2 - A_3 - A_4 + A_1 \cap A_2 + A_1 \cap A_3 +
\\
+ A_2 \cap A_3 - A_1 \cap A_2 \cap A_3 $, где $N_0$ - общее кол-во человек.
\\
По условию $A_1 = A_2 = A_3 = A_4 = 15$, $  A_1 \cap A_2 = A_1 \cap A_3 = A_2 \cap A_3 = 6 $, $ A_1 \cap A_2 \cap A_3 = 1 $. Подставляя в формулу включений-исключений найдем $N_0$:
\\
$N_0 - 4*15 + 6*6 - 4*2 + 1 = 0$
\\
$N_0 = 31$
\\
\textbf{Ответ. $ N_0 = 31 $ -- кол-во студентов в группе.}
\newpage
\textbf{Ex.3.} Найдите количество функций алгебры логики от n переменных, существенно зависящих от всех своих переменных.
\\

Общее кол-во функций от n переменных - $N_0 = 2^{2^n}$. Нужно отнять из них те, что содержат фиктивные переменные. То есть если $A_i$ - множество функций, в которых фиктивна $x_i$, то искомое кол-во функций:
\\
 $ N(\overline A_1, \overline A_2, ... , \overline A_i, ... , \overline A_n) = N_0 - (A_1 \cup A_2 \cup ... \cup A_i \cup ... \cup A_n)$ находится по формуле включений-исключений.
\\
Ясно, что во множестве $A_i$ одна переменная зафиксирована, то есть всего функций -- $ 2^{2^{n-1}} $. Аналогичный ответ для всех A. Тогда всего слагаемых такого вида -- $ C_n^1 = n $
\\
Возьмем пересечение двух A. В этом случае имеем две зафиксированные переменные. Всего функций -- $2^{2^{n-2}}$. Аналогично все слагаемые такого типа будут одинаковыми, всего их -- $ C_n^2 $.
\\
Аналогично для остальных пересечений. Каждое новое кол-во множеств в пересечении уменьшает значение в степени и увеличивает аргумент в кол-ве сочетаний: $ C_n^k * 2^{2^{n-k}} $. Так же по формуле включений-исключений каждое нечетное слагаемое (без учета $N_0$) имеет знак минус. Составим общую формулу полученной суммы:
\\
$N = 2^{2^{n}} + \sum\limits_{k=1}^n (-1)^{k} * C_n^k * 2^{2^{n-k}} = \sum\limits_{k=0}^n (-1)^{k} * C_n^k * 2^{2^{n-k}}$
\\  
\textbf{Ответ. $ \sum\limits_{k=0}^n (-1)^{k} * C_n^k * 2^{2^{n-k}} $ функций.}
\\

\textbf{Ex.4.} Сколько существует шестизначных чисел, к записи которых
есть единица и двойка?
\\

Пусть множество $A_1$ -- кол-во шестизначных чисел, не содержащих единицу, $A_2$ - кол-во шестизначных чисел, не содержащих двойку. Искомое множество: $N(\overline A_1, \overline A_2) = N_0 - A_1 - A_2 + A_1 \cap A_2$
\\
$ A_1 = 8*9*9*9*9*9 $, $ A_2 = 8*9*9*9*9*9 $, $ A_1 \cap A_2 = 7*8*8*8*8*8$
\\
$ N_0 = 9*10^5 $
\\
$N(\overline A_1, \overline A_2) = 9*10^5 - 8*9^5 - 8*9^5 + 7*8^5 = 184592$
\textbf{Ответ. 184592 числа.}
\end{document}
