\documentclass[a4paper,12pt]{article}

\usepackage[T2A]{fontenc}
\usepackage[utf8]{inputenc}
\usepackage[english.russian]{babel}
\usepackage{graphicx}
\graphicspath{{pictures/}}
\DeclareGraphicsExtensions{.pdf,.png,.jpg}

\usepackage{amsmath,amsfonts,amssymb,amsthm,mathtools}

\author{Altunov A.}
\title{АЛКТГ - 4 задание}
\date{\today}

\begin{document}

\maketitle
\newpage
\textbf{Ex.1.} Существует ли граф на 8 вершинах, в котором 23 ребра и есть вершина степени 1?
\\
Пусть такой граф существует. В нем одна вершина, висячая, имеет всего одно ребро. Тогда имеем в данном графе подграф из $8 - 1 = 7 $ вершин, в котором $ 23 - 1 = 22 $ ребра.
\\
Заметим, что любой граф может иметь максимально $ \dfrac{n*(n-1)}{2} $ ребер. Такой граф является полным. 

Подграф из 7 вершин может иметь максимум $ \dfrac{7*6}{2} = 21 $ ребро. По условию этот граф имеет 22 ребра, противоречие.

Следовательно, указанный граф \textbf{не} может существовать.
\\
\\
 
\textbf{Ex.2.} В шахматном турнире по круговой системе с пятью участниками только Ваня и Леша провели одинаковое число встреч, а все остальные -- разное. Сколько встреч сыграли Ваня и Леша?

Пусть x - кол-во сыгранных Ваней или Лешей встреч, a, b, c -- кол-во сыгранных встреч других участников. Так как участники играли друг с другом, каждая встреча про подсчете учитывается дважды, т.е. $2x + a + b + c = 2k $

Очевидно, слагаемое $ a + b + c $ -- четно. Это может выполняться в двух случаях:

а) Среди слагаемых ровно два нечетных

б) Все слагаемые -- четные  
\\
Рассмотрим каждый случай

а) По условию $ a \neq b \neq c $ Пусть a, b - нечетны. Каждое слагаемое может принимать значение из набора $ {1, 3} $. Слагаемое не может быть >4, так как максимум один человек может сыграть 4 партии. Пусть $ a = 1, b = 3 $.
\\
с может принимать одно из значений $ {0, 2, 4} $. Заметим, что максимально может быть $ \dfrac{5*4}{2} = 10 $ встреч. Пусть с = 4.
\\ 
Чтобы выполнялось ограничение $ 2x + a + b + c \leq 20 $, x = 2. Данный пример не противоречит условиям, следовательно, является решением. 
\\
Пусть с = 2. Тогда x = 0. Участник b сыграл 3 партии, но среди участников двое не играли вообще, противоречие. Если x = 4. Данный пример нарушает ограничение.
\\
Пусть с = 0. Тогда x = 2. При таком набора противоречий нет, следовательно, он правильный.

б) Все слагаемые -- четные. Очевидно, что среди a,b,c есть четверка, иначе ограничение не выполняется.
\\
Тогда имеем набор $ a = 0, b = 2, c = 4 $. Тогда $ 2x \leq 2 $. Т.е. x = 1.
\\
Заметим, что не может одновременно быть значений 4 и 0, это приводит к противоречию (ранее доказано).

Таким образом, единственный корректный случай выполняется при x = 2.

\textbf{Ответ. 2 встречи}
\\
\\
\textbf{Ex.3.} Докажите, что вершины связного графа G можно упорядочить так, что для каждого i, $ 1 \leq i \leq |V(G)| $, индуцированный подграф $ G[{v_1, ... , v_i}] $ будет связным.

Покажем решение с конца. Пусть мы имеем связный граф G  на V(G) вершинах. Будем по порядку убирать вершины, сохраняя связность графа.
\\
По теореме в связном графе можно убрать вершину, сохраняя его связность. Для этого в изначально связном графе удаляем либо висячую вершину (если это остовное дерево), либо вершину, входящую в цикл. Таким образом мы получим связный подграф. Гарантируется, что очередной подграф либо дерево, либо содержит цикл => мы можем удалять вершины до тех пор, пока не дойдем до единичного графа. Запоминая порядок удаляемых вершин, восстановим его после последнего удаления, инверсируя. Так мы получим порядок добавляемых вершин, при которых подграф является связным \textbf{ч.т.д.}
\\
\\
\textbf{Ex.4.} Докажите, что $ rad(G) \leq diam(G) \leq 2*rad(G) $, и приведите примеры, когда достигается каждая из этих оценок.

По определению радиус графа -- минимальный эксцентриситет, а диаметр -- максимальный эксцентриситет. В одном графе невохможно неравенство, в котором максимальное значение меньше минимального. Таким образом, доказываем неравенство $ rad(G) \leq diam(G) $ с частным случаем $ rad(G) = diam(G) $. Рассмотрим его.

Самый простой пример - граф из 2 вершин, соединенных ребром. Здесь есть только один эксцентриситет, и он равен единице. Получаем равенство радиуса и диаметра.


Рассмотрим второе неравенство: $ diam(G) \leq 2*rad(G) $
\\
Пусть оно неверно. Тогда можно найти диаметр, больший 2 радиусов. Но в любом графе минимальный эксцентриситет лежит в центре (или около центра) максимального центриситета (если допустить обратное, любая вершина, не лежащая на диаметре, при проходе до самой дальней вершины проходит через весь диаметр, тем самым не являясь минимальным эксентриситетом, противоречие). Тогда в общем случае получаем, что $ Rad = \lfloor Diam/2 \rfloor $. В переводе на неравенство мы получаем искомую форму $ diam(G) \leq 2*rad(G) + 1 $. Частный случай равенства выполняется в любом линейном графе с нечетным кол-вом вершин.
\\
\\
\textbf{Ex.5.} В дереве на 2019 вершинах ровно три вершины имеют степень 1. Сколько вершин имеют степень 3?

Рассмотрим типы вершин степени 3. Если все ребра из этой вершины уходят к листьям, эта вершина -- старшая. Каждая такая вершина дает минимум 3 листа. Пусть мы имеем n старших вершин. Они дают 3n листьев. По условию имеем 3 листа. $ 3n = 3 $ => n = 1, т.е. старшая вершина степени 3 может быть только \textbf{одна}.
\\
Пусть вершина степени 3 имеет одно ребро от старшей вершины и два к листьям. Тогда n таких вершин дадут минимум 2n листьев. $ 2n = 3 $ При таком равенство можем иметь максимум одну вершину степени три. Таким образом, в любом месте вершина степени три может быть лишь \textbf{одна}.
\\
\\
\textbf{Ex.6.} Есть два дерева на n вершинах, каждое имеет диаметр длины d. Можно ли так добавить ребро между вершинами этих деревьев, чтобы длина диаметра полученного дерева равнялась d?

Пусть искомый диаметр достигается. Соединив два графа ребром, получим связный граф, в котором можно из любой точки добраться в любую. Тогда из любой точки одного диаметра можно добраться до любой точки другого. Возьмем точки на обоих диаметрах вблизи центра. Расстояние от крайней вершины подграфов до этих точек как минимум равно $ \lceil d/2 \rceil $. Между этими вершинами, по нашему определению, можно построить путь, причем этот путь - ненулевой. Таким образом, если построить путь от крайней вершины первого подграфа через серединную вершину по пути ко второй серединной вершине и до крайней вершины второго подграфа, получим следующую длину: $ 2 * \lceil d/2 \rceil + x $, где x=1,2,3.. При любых d данная оценка больше диаметра => допущение неверно, невозможно построить граф с диаметром d.
\\
\\
\textbf{Ex.7.} Докажите, что для любого $ k \leq |V(G)| $. В графе G найдется k вершин $ \lbrace v_i \rbrace ^k_i $ в результате удаления которых вместе со всеми смежными ребрами, получается связный граф $ G' = G[V \backslash \lbrace v_i \rbrace ^k_i ]$ 

Алгоритм удаления аналогичный Ex.3. : в связном графе удаляем висячие вершины или вершины в цикле, тем самым не нарушая связность. Делая это последовательно до требуемого k удаленных вершин, добьемся искомого результата. В итоге получается связный граф, который содержит вершины, не удаленные последовательно (т.е. с индексами, не входящими в интервал 1-k). Так же можно отметить, что одновременно можно удалять все висячие вершины в графе, получая все так же связный граф. Так же в каждом цикле в графе можно удалять 1 вершину одновременно.
\\
\\
\textbf{Ex.8.} Граф получен из графа-цикла $ С_2n $ добавлением ребер, соединяющих противоположные вершины ( $ v_1 $ соединена с $ v_{n+1} $, $ v_2 $ с $ v_{n+2} $ и т.д.). При каких n получившийся граф правильно раскрашиваемый а) в два цвета; б) в три цвета?

а) При четных n: все вершины графа по кругу раскрашиваются поочередно в два цвета. Но при этом противоположные вершины оказывается одного цвета, раскраска неверна. И наоборот, раскрашиваем противоположные вершины в разные цвета. При этом имеем n пар противоположных вершин, и две пары оказывается рядом одинаково раскрашены, что противоречит условию правильной раскраски.

При нечетных n: все вершины графа по кругу раскрашиваются поочередно в два цвета. Соответственно, все противоположные вершины оказываются различно раскрашенными, что и требуется в задаче => искомое n=2k+1, где k=0,1,2...

б) Воспользуемся уже готовой раскраской для 2 цветов. Мысленно поделим граф на две половины по n вершин. В одной половине перекрашиваем центральную вершину в новый третий цвет. Так мы не нарушаем правила, добиваемся правильной закраски

При четных n: в одной половине перекрашиваем крайние с каждой стороны вершины в новый цвет. Таким образом мы решим проблему идентичности пар противоположных закрасок. То есть выполним правильную закраску. Данное правило работает при n>2 (при n=2 между вершинами, закрашенными в третий цвет, нет других вершин)
\\
\\
\textbf{Бонусная задача.} 

\end{document}
