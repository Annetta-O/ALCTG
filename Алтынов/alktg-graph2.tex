\documentclass[a4paper,12pt]{article}

\usepackage[T2A]{fontenc}
\usepackage[utf8]{inputenc}
\usepackage[english.russian]{babel}
\usepackage{graphicx}
\graphicspath{{pictures/}}
\DeclareGraphicsExtensions{.pdf,.png,.jpg}

\usepackage{amsmath,amsfonts,amssymb,amsthm,mathtools}

\author{Altunov A.}
\title{АЛКТГ - 5 задание}
\date{\today}

\begin{document}

\maketitle
\newpage
\textbf{Ex.1.} На танцы пришли n девушек и n юношей. Каждый юноша знаком с двумя девушками, а каждая девушка знакова с двумя юношами. Докажите, что собравшихся можно разбить на n смешанных пар так, чтобы в каждой паре юноша и девушка были знакомы.
\\
\\
Представим знакомых юношей и девушек в графе. Девушек представим вершинами красного цвета, а юношей - синего. Ребра между вершинами обозначают, что пара знакома. По условию каждый человек имеет две связи, т.е. каждая вершина графа имеет степень 2. Следовательно, граф является циклом. Заметим, что при этом граф - правильной раскраски, то есть вершины в нем чередуются по цвету (имеем 2 цвета + 2n вершин - четное). Значит, достаточно удалить ребра графа через одно, при этом оставшиеся пары гарантированно будут разного цвета, что и требуется в задаче.
\\
\\
\textbf{Ex.2.} Докажите, что любое дерево является двудольным графом.
\\
\\
Расположим произвольное дерево по уровням, где первому уровню соответствует корневая вершина, а каждому следующему - дочерние вершины предыдущего. В любом дереве гарантируется, что путь между любыми вершинами единственен. Тогда расположим все четные уровни вершин в одной доле, а все нечетные - во второй. Т.к. для любой вершины невозможно с помощью \textbf{одного} ребра перебраться на уровень, больший или меньший соседнего, все вершины в полученных долях имеют только ребра между долями, что и является условием для двудольного графа.
\\
\\
\textbf{Ex.3.} В квадратной таблице $ N \times N $ записаны неотрицательные числа
так, что сумма в любой строке и в любом столбце равна 1. Докажите,
что в этой таблице можно выбрать N положительных чисел, никакие
два из которых не будут находиться в одной строке или столбце.
\\

Рассмотрим крайний случай, при котором все положительные числа в таблице равны единице. Очевидно, что при выполнении данного варианта, остальные выполняются автоматически.

Представим таблицу в виде двудольного графа 2n, где в одной доле находятся вершины, обозначающие строки, а в другой -- вершины, обозначающие столбцы. Ребра между вершинами обозначают наличие единицы в данной ячейке -- пересечении столбца и строки.
\\
Чтобы выполнялось условие, что каждая строка и столбец имеет сумму один, нужно поставить ребра так, чтобы каждая вершина имела степень 1.
\\
В итоге имеем n ребер между долями. Любое ребро исходит и приходит в различные вершины. Таким образом, в одной строчке или в одном столбце не найдется больше 1 единицы. То есть, мы имеем n единиц, каждая из которых находится в уникальной строчке и в уникальном столбце, что и требуется в задаче.
\\
\\
\textbf{Ex.4.} По кругу записаны 7 натуральных чисел. Известно, что в каждой паре соседних чисел одно делится на другое. Докажите, что найдется пара и несоседних чисел с таким же свойством.
\\

Представим круг с числами в виде орграфа, где входящая вершина является делителем исходящей.


Возьмем три произвольные соседние вершины графа: $  v_1 , v_2 , v_3 $ . Если по ребрам можно прийти из $ v_1 $ в $ v_3 $ или из $ v_3 $ в $ v_1 $ , то $ v_3 $ является делителем $ v_1 $ (или, соответственно, наоборот), что и требуется. То есть если направление двух соседних ребер совпадает, то задача выполняется.


Пусть искомое сочетание не найдется, и все соседние ребра направлены по разному. Имеем 7 вершин по условию, следовательно, 7 ребер между ними.
Чередуя направления: $ \rightarrow , \leftarrow , \rightarrow , \leftarrow , \rightarrow , \leftarrow , \rightarrow $. При этом первое и последнее ребро имеет одинаковое направление, противоречие. Легко увидеть, что при нечетном кол-ве ребер невозможно подобрать направление, удовлетворяющее нашему условию $ \Rightarrow $ в любом кругу из 7 чисел найдется несоседняя пара, в которой одно число делится на другой, ч.т.д.
\\
\\
\textbf{Ex.5.} В шахматном турнире, в котором каждый участник должен был встретиться с каждым, один шахматист заболел и не доиграл свои партии. Всего в турнире было проведено 24 встречи. Сколько шахматистов участвовало в турнире, и сколько партий сыграл выбывший участник?
\\
\\
Пусть в турнире участвовало x человек, а выбывший участник не успел сыграть a партий.
\\
Тогда общее число партий равно: $ \frac{x*(x-1)}{2} - a = 24 $
\\
\\
Преобразуем: $ x^2 - x - 2a = 48 \; \Leftrightarrow \; x^2 - x - (2a + 48) = 0 $
\\
\\
Дискриминант данного квадратного уравнения равен: $ D = 1 + 8a + 192 = 193 + 8a $
\\
Так как число x должно быть целом, дискриминант должен является квадратом целого. При любом а D -- нечетно, ближайшее к 193 сверху нечетное число, являющееся квадратом -- $ 225 = 15^2 $
\\
Тогда $ a = \frac{225 - 193}{8} = 4$ , и $ x = \frac{1 \pm 15}{2}  = 8 $ (второй корень меньше нуля)
\\
Заметим, что следующие нечетные числа, являющиеся квадратами, не удовлетворяют условию. Например, при $ D = 17^2 = 289 $ число $ a = 12 $, а $ x = 9 $. Кол-во человек меньше, чем а, противоречие. Т.к. а возрастает быстрее, чем х, все последующие варианты также не подходят.
\\
Имеем $ x = 8, a = 4 $. Выбывший шахматист сыграл $ x - a = 4 $ партии.
\\
\\
\textbf{Ответ. 8 человек, выбывший участник сыграл 4 партии}
\\
\\
\textbf{Ex.6.} В некоторой стране есть столица и еще 100 городов. Некоторые города (в том числе и столица) соединены дорогами с односторонним движением. Из каждого нестоличного города выходит 20 дорог, и в каждый такой город входит 21 дорога. Докажите, что в столицу неьзя проехать ни из одного города.
\\
\\
Предположим, что в столицу все же есть пути. Кол-во ребер, направленных в вершину-столицу, обозначим за х, а кол-во ребер, выходящих из вершины-столицы -- за y.
\\
Тогда всего входящих в вершины ребер: $ a = 100 \cdot 21 + x $
\\
Всего ребер, выходящих из вершин: $ b = 10 \cdot 20 + y $
\\
Так как любое ребро, выходящее из одной вершины, является входящим для другой: $ a = b $
\\
Кроме того, так как кроме столицы есть всего 100 вершин, а все пути - односторонние, можно установить ограничение на x,y: $ x + y \leq 100 $
\\
Имеем: 
\begin{equation*}
 \begin{cases}
100 \cdot 21 + x = 10 \cdot 20 + y 
\\
 x + y \leq 100 
 \end{cases}
\end{equation*}
\\
Преобразуем:
\\
\begin{equation*}
 \begin{cases}
y - x = 100
\\
y \leq 100 - x
 \end{cases}
\end{equation*}
\\
Что приводит к нер-ву: $ 100 - 2x \geq 100 \; \Leftrightarrow \; 2x \leq 0 $\\
$ x = 0 $, т.е. в столичной вершине нет ни одного входящего ребра, ч.т.д.
\end{document}
