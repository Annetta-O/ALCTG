\documentclass[a4paper,12pt]{article}

\usepackage[T2A]{fontenc}
\usepackage[utf8]{inputenc}
\usepackage[english.russian]{babel}
\usepackage{graphicx}
\graphicspath{{pictures/}}
\DeclareGraphicsExtensions{.pdf,.png,.jpg}

\usepackage{amsmath,amsfonts,amssymb,amsthm,mathtools}

\author{Altunov A.}
\title{АЛКТГ - 2 задание}
\date{\today}

\begin{document}

\maketitle
\newpage
\textbf{Ex. 1.} 
 a) Под условия подходят числа вида $ 10^k, где k \in [0,2] $
\\
\\
$ A = \lbrace x \; | \;  x = 10^k, \; 0 \leq k \leq 2, \;  k \in R \rbrace $
\\
\\
б) В множество входят все действительные числа, большие 5:
\\
\\
$ B = \lbrace x \;|\;  x > 5, \;  x \in R \rbrace $
\\
\\
в) В множестве содержатся все \underline{натуральные} числа, меньшие, чем 5:
\\
\\
$ C = \lbrace x \; | \; x < 5, \; x \in N \rbrace $
\\
\\
г) В множестве не содержатся дествительные числа, для этого можно сделать невыполнимое для их наличия условие:
\\
\\
$ \lbrace x \; | \; x^2 + 5 = 0, \; x \in R \rbrace $
\\
\\
\textbf{Ex 2.} 
 $(A \cup B) \backslash (A \cap B)=(A \backslash B) \cup (B \backslash A)$
\\
\\
Представим левое выражение в виде:
$ (A \vee B )\wedge \neg (A \wedge B) $
\\
\\
Преобразем:
\\
\\
$ (A \vee B )\wedge \neg (A \wedge B) \; = \; (A \vee B )\wedge (\neg A \vee \neg B) \; = \; (A \vee B) \wedge \neg A \vee (A \vee B) \wedge \\ \wedge \neg B \; = \; (A \wedge \neg A) \vee (\neg A \wedge B) \vee (A \wedge \neg B) \vee (B \wedge \neg B) \; = \;(\neg A \wedge B) \vee (A \wedge \neg B)$
\\
\\
Последнее выражение эквивалентно данному виду: $ (A \backslash B) \cup (B \backslash A) $
, \; ч.т.д
\\
\\
б) Представим выражение в виде: $ (A \vee B) \wedge \neg C $
\\
\\
Преобразуем:
\\
\\
$ (A \vee B) \wedge \neg C = (A \wedge \neg C) \vee (B \wedge \neg C) $
\\
\\
Последнее выражение эквивалентно виду: $ (A \backslash C) \cup (B \backslash C)$, ч.т.д.
\\
\newpage
\textbf{Ex. 3.}
Заметим, что если подмножество, входящее в следующее множество -- меньше, то диаграмма пересечения выглядит, как на рис 1.
\\
При этом $ A_1 \backslash A_4 $ --  Круг $ A_1 $ без $ A_4 $ , содержит $ A_2 $ , $ A_3 $ без  $ A_4 $
\\
\\
%\includegraphics[scale=0.15]{123}
\\
\\
Однако при этом множества $ A_6 $ , $ A_9 $ , являющиеся подмножствами множества $ A_4 $ , имеют размер гораздо меньший, чем $ A_1 \backslash A_4 $, т.к. они меньше $ A_4 $ . То есть$ A_6 \backslash A_9 $ входит в $ A_4 $, которое вычитается из $ A_1 $ . Таким образом, исходное равенство не может быть выполнено
\\
Только при условии (рис 2) исходное условие будет выполнено, так как, $ A_1 \backslash A_4 $ -- это пустое множество, так же, как и $ A_6 \backslash A_9 $ 
\\
\\
При учете этого:
\\
\\
$ A_2 \backslash A_7 = \varnothing$
\\
$ A_3 \backslash A_8 = \varnothing$
\\
=> $ A_2 \backslash A_7 = A_3 \backslash A_8 $ \; ч.т.д
\\
\\
\textbf{Ex. 4.} Пусть A = <<$ \sqrt3 $ - иррационально>> , B = <<$ \sqrt2 $ - иррационально>>, C = <<$ \sqrt2 + \sqrt3 $ - иррационально>>
\\
\\
Тогда надо доказать: $(A \vee B) \longrightarrow C = 1$
\\
\\
$(A \vee B) \longrightarrow C = \neg A \vee \neg B \vee C$
\\
\\
$ \sqrt3 $ - рационально, или $ \sqrt2 $ - рационально, или $ \sqrt2 + \sqrt3 $ - иррационально
\\
\\
Докажем иррациональность $ \sqrt3 $ и $ \sqrt2 $ :
\\
Пусть $ \sqrt3 $ - рационально. Тогда его можно представить в виде: $ \sqrt3 = \frac{m}{n}$
\\
\\
$ 3 = \left( \frac{m}{n} \right)^2 $, $ 3n^2 = m^2 $
\\
\\
Если представить левую и правую части как произведение простых множителей с некоторыми коэффициентами, то эти части будут иметь разную четность степени тройки (слева -- нечетная, справа -- четная) => равенство не может быть выполнено - противоречие. Значит, \underline{число $ \sqrt3 $ - иррационально}. Иррациональность $ \sqrt2 $ доказывается аналогично.
\\
\\
Следовательно, оба числа являются иррациональными, а значит, чтобы выражение $ \neg A \vee \neg B \vee C$ было истинно, С должно быть иррациональным, ч.т.д.
\\
\\
\textbf{Ex. 5.} Исходную сумму можно представить как сумма рядов сумм:
\\
\\
$
\left.
  \begin{array}{ccc}
2 + 2^2 + 2^3 + ... + 2^n 
\\
2^2 + 2^3 + ... + 2^n 
\\
2^3 + ... + 2^n 
\\
...
\\
2^n
\end{array}
\right\}
$
n строк
\\
\\
Сумма каждой строки - сумма геометрической прогрессии с различным первым членом и кол-вом членов:
\\
\\
$ S1 = \frac{2(2^n-1)}{2-1} = 2^{n+1} - 2 $
\\
\\
$ S2 = 2^2(2^{n-1}-1) = 2^{n+1} - 2^2 $
\\
\\
...
\\
\\
$ S3 = 2^n(2^1-1) = 2^{n+1} - 2^n$
\\
\\
Сложив все суммы, получим: 
\\
\\
$ S = (2^{n+1} + 2^{n+1} + ... + 2^{n+1}) - (2 + 2^2 + 2^3 + ... + 2^n) $
\\
\\
$ S = n2^{n+1} - 2(2^n - 1) = (n-1)2^{n+1} + 2$, \textbf{ч.т.д.}
\\
\\
\textbf{Ex. 6.} Докажем утверждение с помощью математической индукции по n.
\\
\underline{База индукции.} При n = 1 имеем прямоугольник 1х3 с тремя разными цветами в единственном столбце. База доказана.
\\
\underline{Переход индукции.} Пусть в левой верхней клетке прямоугольника находится фишка с цветом а. Имеем два варианта:
\\
\\
1) Вторая строчка прямоугольника состоит из фишек цвета а и фишек б. Тогда в третьей строчке минимум одна фишка цвета в (т.к. фишек а всего n, 1 уже занята, остальные не смогут полностью покрыть третью строчку). Эту фишку мы ставим на первую позицию третьей строчки.
\\
\\
2) Вторая строчка содержит и а, и б, и в. Третья строка при этом точно содержит цвет, отличный от а, так как всего n фишек с цветом а. Ставим эту фишку на первую позицию. Тогда во второй строке переставляем фишку с отличным от третьей и первой фишек цветом.
\\
\\
Таким образом, мы в любом случае получаем столбец с тремя различными цветами. Убираем его из прямоугольника. Остается таблица 3(n-1). По предположению индукции в ней можно составить столбцы с различными цветами
\\
\\
Следовательно, мы доказали возможность переставления фишек в таблице 3n, ч.т.д.
 
\end{document}
