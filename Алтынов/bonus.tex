\documentclass[a4paper,12pt]{article}

\usepackage[T2A]{fontenc}
\usepackage[utf8]{inputenc}
\usepackage[english.russian]{babel}
\usepackage{graphicx}
\graphicspath{{pictures/}}
\DeclareGraphicsExtensions{.pdf,.png,.jpg}
\usepackage[left=2cm,right=2cm,
    top=2cm,bottom=2cm,bindingoffset=0cm]{geometry}
\usepackage{amsmath,amsfonts,amssymb,amsthm,mathtools}

\author{Altunov A.}
\title{АЛКТГ - Бонусные задачи (3, 4, 7)}
\date{\today}

\begin{document}

\maketitle
\newpage
\textbf{Week 3.}
\\

Чтобы функция от двух переменных являлась полной, она не должна принадлежать ни одному из предполных классов: $ T_0, T_1, M, S, L $.

$ f(x,y) \nsubseteq T_0$, $ \Rightarrow f(0,0) = 1$

$ f(x,y) \nsubseteq T_1$, $ \Rightarrow f(1,1) = 0$

$ f(x,y) \nsubseteq M $, выполняется автоматически из предыдущих выражений.

$ f(x,y) \nsubseteq S $, $ \Rightarrow f(x,y) \neq \overline{f}(\overline{x},\overline{y}) $

$ f(0,1) \neq \overline{f}(1,0) \Rightarrow  f(0,1) = f(1,0) $

С учетом всех этих выражений, получаем два возможных вектора значений функции f: [1,0,0,0] -- стрелка Пирса, [1,1,1,0] -- штрих Шеффера.
\\
Проверим обе функции на принадлежность классу L.

Пусть $f_2 = [1,0,0,0] \in L$. Тогда ее можно представить в виде: $ f(x,y) = a_0 + a_1 x + a_2 y $
\\
$f(0,0) = 1 = a_0 + a_1 0 + a_2 0 = a_0 \Rightarrow a_0 = 1 $
\\
$ f(1,1) = 0 = 1 + a_1 1 + a_2 1 $ не выполняется при всех $a_1, a_2$.
$ \Rightarrow f(x,y) \nsubseteq L $.
\\
Доказательство для $f_2 = [1,1,1,0]$ аналогично данному.
Таким образом мы доказали, что функции Стрелка Пирса и Штрих Шеффера являются единственными полными функциями от двух переменных, ч.т.д.
\\

\textbf{Week 4.} 
\\

 Интерпретируем задачу в виде графа: имеем граф с 12 вершинами, обозначающих людей. Пусть ребро в графе между двумя вершинами обознает то, что люда незнакомы.

Если в графе нет циклов нечтеной длины, из него можно сделать двудольный граф. По определению двудольного графа вершины каждой доли не соединены между собой. В каждой доле 6 вершин, значит гарантировано найдется $ 6/2*2 = 6 $ пар несвязанных вершин, ч.т.д.
  Предположим, что в графе есть циклы нечётной длины. Рассмотрим нечётный цикл минимальной длины. Рассмотрим все возможные случаи.

  1) Цикл длины 3. Тогда, если среди девяти человек, не входящих в этот цикл, есть двое незнакомых, то среди оставшихся семи человек из каждых четырёх найдутся трое знакомых (добавим к этим четырём двух незнакомых из девятки и трёх незнакомых из цикла).Значит, в подграфе на семи вершинах каждые два ребра имеют общую вершину. Любое третье ребро проходит через эту вершину, иначе среди четырёх человек не найдутся трое знакомых. Таким образом, все ребра имеют общую вершину, и, удаляя эту вершину, мы получаем шесть попарно знакомых.

  2) Цикл длины 5. По аналогии с (1), среди оставшихся семи из каждых четырёх найдутся трое знакомых, и среди этих семи найдутся шесть знакомых.

  3) Цикл длины 7. Тогда для пяти человек, не входящих в этот цикл, все попарно знакомы (если есть двое незнакомых, то, добавив к ним семерых из цикла, получим противоречие). Если есть человек из цикла, знакомый со всеми этими пятью, то условие выполняется. Иначе каждый из цикла не знаком с кем-то из оставшихся. Так как  $ 7 > 5 $,  то найдётся человек из оставшихся, не знакомый с двумя из цикла (по принципу Дирихле)
  Из того, что мы взяли нечётный цикл минимальной длины, следует, что 2 человек должны быть "незнакомы через одного".
  Но тогда этот человек знаком со всеми из пяти оставшихся, потому что удаляя из цикла вершину, мы получаем снова цикл длины 7, а в дополнении к циклу длины 7 все попарно знакомы.

 4) Цикла длины 9 нет по условию.

  5) Цикл длины 11. Тогда по аналогии с (3) оставшийся человек может быть не знаком максимум с двумя из цикла. Но тогда в цикле легко найти пять человек, знакомых между собой и с оставшимся, \textbf{ч.т.д.}
\\

\textbf{Week 7.}
\\

Всего сочетаний с повторений из n различных букв по n: $ C_{n+n-1}^n = C_{2n-1}^n $
\\
Всего букв в сочетаниях: $ C_{2n-1}^n * n $. Каждая буква повторяется одинаковое кол-во раз, иначе не выполнялась бы симметрия. 
\\
То есть для каждой буквы число повторений: $ C_{2n-1}^n * n / n = C_{2n-1}^n $, ч.т.д.  
\end{document}
