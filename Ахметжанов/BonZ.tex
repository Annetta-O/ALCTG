\documentclass[a4paper,11pt]{article}

\usepackage[T2A]{fontenc}
\usepackage[normalem]{ulem}
\usepackage[utf8]{inputenc}
\usepackage[english,russian]{babel}
\usepackage{multirow}
\usepackage{array}
\usepackage{amsmath,amsfonts,amssymb,amsthm}

\author{Ахметжанов Руслан}
\title{Некоторые бонусные задачи}
\date{\today}

\begin{document}
\maketitle 


\section*{I неделя}
 \textbf{Найдите асимптотическую оценку количества ,булевых функций от n переменных, которые зависят от всех своих аргументов существенно. Иначе говоря, надо придумать такие верхнюю и нижнюю оценки на это количество, чтобы их отношение стремилось к 1 при n $ \rightarrow \infty$}
 
 	Для начала оценим количество функций у которых существуют фиктивные переменные:
 Если есть одна фиктивная переменная, то можно рассматривать, как функцию от (n - 1) переменных, а таких всего $2^{2^{n-1}}$(как и зависящих от всех своих аргументов существенно, так и с фиктивными), так как фиктивной может быть n переменных, то наша оценка сверху -- $n*2^{2^{n-1}}$. Заметим, что при больших n отношение нашей оценки к количеству всех функций стремится к нулю: $ \frac{n*2^{2^{n-1}}}{2^{2^n}} = n*2^{-2^{n-1}} = \frac{1}{2^{2^{n-1}}}$, а раз количество функций, которые зависят не от всех своих аргументов существенно при больших n пренебрежимо мало, то количество всех функций, то есть $2^{2^n}$, и есть асимптотическая оценка количества булевых функций от n переменных, которые зависят от всех своих аргументов существенно.
\section*{II неделя}
\textbf{Докажите, верность неравенства для k > 3}

\[ \frac{ x_{1}}{ x_{k} +  x_{2}} + \frac{ x_{2}}{ x_{1} +  x_{3}} +...+  \frac{ x_{k}}{ x_{k - 1} +  x_{1}} \geq 2.\]

\textbf{Используем Метод математической индукции:}

\begin{center}
	База: k=4
\end{center}

\[\frac{ x_{1}}{ x_{4} +  x_{2}} + \frac{ x_{2}}{ x_{1} +  x_{3}} + \frac{ x_{3}}{ x_{2} +  x_{4}} + \frac{ x_{4}}{ x_{3} +  x_{1}} =\frac{ x_{1} + x_{3}}{ x_{4} + x_{2}} + \frac{ x_{2} + x_{4}}{ x_{1} + x_{3}} \geq 2\]

\begin{center}
	Заметим, что в последнем представлении слагаемые взаимнообратные, а сумма взаимнообратных не меньше двух, значит неравенство верно.
\end{center}

\textbf{Шаг: пусть верно для k, тогда проверим для n+1:}
\begin{flushleft}
	Так как при изменении индексов иксов по кругу сумма не изменится, то можно за минимальное число принять $x_{k+1}$, тогда:
\end{flushleft}

\[ \frac{ x_{1}}{ x_{k + 1} +  x_{2}} + \frac{ x_{2}}{ x_{1} +  x_{3}} +...+  \frac{ x_{k}}{ x_{k - 1} +  x_{k + 1}} + \frac{ x_{k + 1}}{ x_{k} +  x_{1}} > \]
\[ > \frac{ x_{1}}{ x_{k} +  x_{2}} + \frac{ x_{2}}{ x_{1} +  x_{3}} +...+  \frac{ x_{k}}{ x_{k - 1} + x_{1}} + 0 \geq 2 \], что и требовалось доказать.
 

 \section*{III неделя}
 \textbf{Докажите, что штрих Шеффера и стрелка Пирса — единственные функции от двух переменных, через которые выражаются все функции алгебры логики.}
 
\textbf{a)} 
Докажем, что предлагаемые базисы действительно полны:
Так как мы знаем, что базис из коньюкции, дизъюнкции и отрицания -- полный, то из 

$ x|x = \bar{x} = x \downarrow x$

 $(x|x)|(y|y) = x \vee y =(x \downarrow y)\downarrow (x \downarrow y)$
 
 $(x|y)|(x|y) = x \wedge y = (x \downarrow x)\downarrow(y \downarrow y)$
 
следует, что и рассматриваемые нами базисы тоже полны.
 
\textbf{b)} 
Докажем их единственность:

 По теореме Поста cитема булевых функций F является полной тогда и только тогда, когда она целиком не принадлежит ни одному из замкнутых классов. То есть в первой строке таблицы истинности значениt <<1>>, а в последней <<0>>. Рассмотрим все возможные такие функции, их $2^2 = 4 :$
 \begin{tabular}{ | l | l | l | l | l | l | l |}
\hline
$x_1$ & $x_2$ & $f(x_1, x_2)$ & $f_1(x_1, x_2)$ & $f_2(x_1, x_2)$& $f_3(x_1, x_2)$ & $f_4(x_1, x_2)$\\ \hline
0 & 0 & \multicolumn{5}{|c|}{1} \\\hline
0 & 1 & ? & 1 & 0 & 0 & 1\\ \hline
1 & 0 & ? & 0 & 1 & 0 & 1\\ \hline
1 & 1  & \multicolumn{5}{|c|}{0} \\\hline
\end{tabular}
 
 Функции $f_1$ и $f_2$( $f_1 = 1 \oplus x \oplus(y*0), f_2 = 1 \oplus y \oplus (x*0)  $ ) -- линейные, то есть не удовлетворяют теореме Поста и соответственно задаче, а $f_3, f_4$ -- cтрелка Пирса и штрих Шеффера соответственно. Значит из функций от двух переменных только стрелка Пирса и штрих Шеффера -- единственные функции от двух переменных, базис из которых полон, что и требовалось доказать.

 \section*{IV неделя}
 \textbf{В некоторой группе из 12 человек среди каждых 9 найдутся 5 попарно знакомых. Докажите, что в этой группе найдутся 6 попарно знакомых.}
 
 Рассмотрим граф знакомств, в котором 12 вершин -- люди, а рёбра означают отсутствие знакомства. Так как в граф без циклов нечётной длины --двураскрашиваемый, то если в нашем графе нет циклов нечётной длины, то граф можно разделить на две части, в которых вершины не соединены и найдутся 6 попарных знакомых. Предположим, что в нашем графе присутствуют циклы нечётной длины, рассмотрим минимальный такой цикл(все 5 вариантов):
 
 3)  Тогда, если среди девяти вершин не из этого цикла есть две несоединённых вершины, то в каждых 4 вершинах из остальных семи будет три несоединённых, а это значит, что в таком подграфе для каждых двух рёьер будет общая вершина и любое третье ребро пройдёт через эту вершину, так как в другом случае среди четырёх вершин не найдётся трёх несвязанных. Поэтому все рёбра имеют общую вершину, удалив эту вершину получится 6 попарно знакомых;
 
 5)Тогда, по аналогии с предыдущей длиной, в каждых 4 вершинах из остальных семи будет три несоединённых, и в этих же семи вершинах(людях) найдётся шесть попарно знакомых;
 
 7)Тогда все пять людей(вершин) не из цикла попарно знакомы. Всё хорошо, если в цикле есть вершина несоединённая со всеми пятью вершинами(как раз выполнение условия), иначе каждая вершина соединена с какойлибо другой, так как длина цикла больше 5, то существует вершина(*) такая, что она не соеденина с двумя из цикла, поскольку мы рассматриваем наименьший цикл, то между двумя этими вершинами \textbf{в цикле} находится третья вершина, тогда учитывая, что при удалении из цикла третьей вершины у нас всё равно останется цикл длины 7, так как вместо удалённой в него войдёт(*), все пять людей(вершин) не из цикла попарно знакомы, поймём, что третья вершина не соединена ни содной вершиной не входящих в рассматриваемый цикл, то есть имеется шесть попарно знакомых;
 
 9)Противоречит условию;
 
 11)Тогда, по аналогии с циклом длины семь, замтно, что вершина не из цикла может быть соединена максимум с двумя вершинами из цикла, и из этого в цикле есть пять несоединённых вершин, таких, что они все несоединенны с вершиной из цикла.


 
 \section*{VII неделя}
 \textbf{Выписаны все сочетания с повторениями из n букв по n. Докажите, что каждая буква встретится ${2n - 1 \choose n}$ раз.}
  
  Всего слов выписанно ${2n - 1 \choose n}$, так как в каждом слове по n букв, то всего букв выписанно $n{2n - 1 \choose n}$. Так как каждая буква равноценна, то есть если рассматривать варианты включений какой-то букы в слова, то результат справедлив для любой другой буквы(порядок не важен, а при замене букв количество слов не меняется), то есть всех букв выписанно одинаковое количество, тогда это количество $\frac{n{2n - 1 \choose n}}{n} = {2n - 1 \choose n}$. Это также заметно, если рассмотреть как включаются буквы, если какая либо буква повторяется, то нет хотя бы стольки же букв в рассматриваевом слове, а так как все бувы равноценны, то существует ситуация, где на оборот столько же раз эта буква не встретилась.
  
 

 \section*{VIII неделя}
  \textbf{Решите в целых числах уравнения:}
  
  
  \textbf{a)} \[x^2 - xy - y^2 = 1\]
  Допустим, что натуральные x и y являются решением уравнения тогда:
  
  $ x^2 - xy - y^2 > 0 $
  
  $ x^2  - y^2 > xy > 0  \Rightarrow y < x \leq 2y$
  
  Тогда изэтого следует, что если есть одно решение, то есть целое семейство решений связанных так: $\exists$ (x, y): решение уравнения $\Rightarrow \exists$ решение уравнения :(2x + y, x + y), Если рассматривать последовательность из $x_n, y_n$,  являющимися решениями уравнения:
  $y_0, x_0, y_1, x_1 ...$, то рекурентная формула дл вычисления какого-то члена последовательности эквивалентна формуле у чисел фибоначи,
  заметим также, что частным решением данного уравнения является пара (1,0), а это как раз и есть первое и нулевое число Фибоначи то есть решением уравнения являются пары вида $(F_n, F_{n-1})$, где $F_n$-числа фибоначи, за исключением того, что у нас коэффиценты от нуля могут не только увеличиваться, но и уменьшаться, и так как у нас разница положительна, то n $\in$ Z,нечётное.

\textbf{b)} \[x^2 - xy - y^2 = -1 \]
  По аналогии с предыдущим пунктом получаем, за исключением знака разности, то есть n -- чётное.

 
  
 \section*{XI неделя}
 \textbf{Докажите, что для простого числа p справедливо соотношение \[ B_{p+n}\equiv B_n+B_{n+1} (modp) \]}
 Рассмотрим задачу о количестве разделений на части множества мощностью (n+p), это количество $B_{p+n}$. 
 Будем говорить о разбиении на части множества {1,2,...,n,n+1,...,n+p}. Число способов это сделать -- $B_{n+p}$. Рассмотрим некую перестановку последних p элементов: может породить, как старое, так и новое разбиение. Первое случится , когда переставили  элементы одной части, или переставили одинаковые по мощности часи. Всего таких перестановок -- p!, а число разбиений с разной перестановкой, то используем формулу перестановокс повторениями, p может сократиться лишь, если все p элементов в одной части, все элемнты с единичными частями, значит в остальных случаях остаток от деления на p -- 0. Но если эти p элементов присутствуют единожды, то таких разбиений $B_{n}$, а если вместе, то $B_{n+1}$/




\end{document}
