\documentclass[a4paper,11pt]{article}

\usepackage[T2A]{fontenc}
\usepackage[utf8]{inputenc}
\usepackage[english,russian]{babel}

\usepackage{amsmath,amsfonts,amssymb,amsthm}

\author{Ахметжанов Руслан}
\title{•}
\date{\today}

\begin{document}

\maketitle 

\section{Сколько существует слов длины n, состоящих из букв a и b таких, что они не заканчиваются на a и две буквы b не стоят рядом?}
 В задаче 6 на восьмую неделю был разобран аналогичный случай, но вместо b там единицы, а вместо a там нули. Но есть и отличие, Если там длина последовательности была $n(16)$, то сейчас $n-2$(два последних места всегда <<ab>>). Тогда ответом на задачу будет $F_{n}$, где $F_n$-- n-ное число фибоначи.
 А из задачи с семинара мы знаем,
 
  что $F_n = \frac{1}{\sqrt{5}}(1+\sqrt{5})^n - (1-\sqrt{5})^n $
 
\section{Упростите $\sum_{k=0}^{m}(-\frac{1}{2}) {m \choose k} {2k \choose k}$} 

$\sum\limits_{k=0}^{m}(-\frac{1}{2})^k {m \choose k} {2k \choose k} = \sum\limits_{k=0}^{m} \frac{(-1)^k}{2^k} \frac{2^k}{k!} {m \choose k} = \sum\limits_{k=0}^{m}  \frac{(-1)^k}{k!} {m \choose k}$

Заметим схожесть со знакопеременной суммой биноминальных коэффицентов:

$ \sum\limits_{k=0}^{m} {m \choose k} (-x)^k = (1-x)^m$

Так как у нас |x| = 1 пусть $g_{(x)} = \sum\limits_{k=0}^{m} {m \choose k}(x)$, 

тогда 
$G_{(x)} = \int g_{(x)} =  \sum\limits_{k=0}^{m} {m \choose k} \frac{(-x)^2}{2} $, $ \int \int g_{(x)} =  \sum\limits_{k=0}^{m} {m \choose k} \frac{(-x)^3}{2*3}$


И так далее, всего k-1 раз, в итоге получаем: $\sum\limits_{k=0}^{m}  \frac{(-x)^k}{k!} {m \choose k}$

А так как  $g_{(x)} = \sum\limits_{k=0}^{m} {m \choose k}(x) = (1-x)^m$,

то  $\sum_{k=0}^{m}(-\frac{1}{2}) {m \choose k} {2k \choose k} = \frac{x^{k-1}}{(k-1)!} - \frac{x^k}{k!} $
\end{document}