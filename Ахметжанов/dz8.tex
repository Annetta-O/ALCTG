\documentclass[a4paper,11pt]{article}

\usepackage[T2A]{fontenc}
\usepackage[utf8]{inputenc}
\usepackage[english,russian]{babel}

\usepackage{amsmath,amsfonts,amssymb,amsthm}

\author{Ахметжанов Руслан}
\title{Задание на восьмую неделю}
\date{\today}

\begin{document}

\maketitle 


\section{Робот ходит по координатной плоскости. На каждом шаге он
может увеличить одну координату на 1 или обе координаты на 2.
Сколько есть способов переместить Робота из точки (0, 0) в точку
(4, 5)?}
 Если брать, что у робота есть возможность только увеличивать любую из координат на 1, то по определению треугольника паскаля в нём будет ответ, но у данного нам робота есть ещё возможность одновременно увеличивать обе координаты на 2, но так как нас интересует количество путей, то данное дествие возможно, как несколько увеличений разных координат, значит в треугольнике паскаля и это учтено.
\begin{tabular}{ | l | l | l | l | l | l |}
\hline
1 & 6 & 21 & 56 & \textbf{126}\\ \hline
1 & 5 & 15 & 35 & 70\\ \hline
1 & 4 & 10 & 20 & 35 \\ \hline
1 & 3 & 6 & 10 & 15\\ \hline
1 & 2 & 3 & 4 & 5\\ \hline
1 & 1 & 1 & 1 & 1\\ \hline
\end{tabular}

\section{В магазине продается 10 видов пирожных. Сколькими спосо-
бами можно купить 100 пирожных (порядок покупки не важен)?}
 Можно представить, что выбранные 100 пирожных мы складываем в 10 коробок в зависимости от вида, тогда у нас есть есть 100 + 10 -1 мест для выбора границ коробки, которых 10 - 1, а ответом будет являться \textbf{${109 \choose 9}$}

\section{Сколько различных слов (не обязательно осмысленных) можно
bcrполучить, переставляя буквы в словах}

\subsection{«ЛИНИЯ»}
 Всего букв в слове 5, тогда первую букву выбрать 5 вариантов, вторую 4 и так далее -- всего 5!, но буква И повторяется два раза,   поэтому разделим на 2!. Конечный ответ:$\frac{5!}{2!}$	

\subsection{«ОБОРОНОСПОСОБНОСТЬ»}
 Всего букв в слове 18, тогда первую букву выбрать 18 вариантов, вторую 17 и так далее -- всего 18!, но буква О повторяется семь раза, буква С повторяется три раза, буква Б повторяется 2 раза, буква Н повторяется два раза, поэтому разделим на 2!7!2!3!.
 Конечный ответ:$\frac{18!}{2!2!3!7!}$	
 
 \section{Ex.4}
 \subsection{Как связаны между собой ${n \choose k}$ и ${n \choose k + 1}$}
 ${n \choose k} = \frac{n!}{k!(n-k)!};$ \\*
 ${n \choose k + 1} =\frac{n!}{(n - k - 1)!(k+1)!} = \frac{\textbf{n!}}{\textbf{k!} (k + 1) \textbf{(n-k)!} \frac{1}{n-k}}= {n \choose k} \frac{n-k}{k+1}$ \\*\\*
 \[{n \choose k + 1} =  {n \choose k} \frac{n-k}{k+1}\]

 \subsection{Какое из слагаемых в разложении $(1 + 2)^n$ по формуле Бинома Ньютона будет наибольшим?}
 
 Формула Бинома Ньютона:
 \begin{center}
	$(a + b)^n = \sum\limits_{k}{n \choose k} a^{n-k} b^k $
\end{center}

Для нашего случая:
\begin{center}
	$(1 + 2)^n = \sum\limits_{k}{n \choose k} b^k $
\end{center}

Так как с уменьшением k  ${n \choose k}=\frac{n!}{k!(n-k)!}$ возрасает, а $2^k$ убывает, то интересующий нас момент неясно где находится, поэтому рассмотрим отношение двух соседних элементов:

Исходя из предыдущего пункта отношение между соседними членами $ \frac{n-k}{k+1} 2 $, сравним с единицей:

\begin{center}
$ \frac{n-k}{k+1} 2 > 1 $ \\
Так как k > 1, то можем домножить на (k + 1)\\
$ 2n - 2k > k + 1 $\\
$ 3k > 2n - 1 $\\
$ k > \frac{2}{3}n - \frac{1}{3} $\\
Это значит, что при $ k > \frac{2}{3}n - \frac{1}{3} $ наши слагаемые увеличиваются, а после убывают(с уменьшением k), и интересующий нас максимум случится при $k = \frac{2}{3}n - \frac{1}{3}$, а само слагаемое: \[{n \choose \frac{2}{3}n}  b^{\frac{2}{3}n} \]
\end{center}

\section{Дайте комбинаторное доказательство тождеств:}

\subsection{${n \choose m}{m \choose k}={n \choose k}{n-k \choose m-k}$} 

Нам необходимо выбрать из n солдат m человек и назначить k офицеров из них. Сколькими способами это можно сделать?

В первом случае мы сначала выибраем m солдат(${n \choose m}$), и уже из них k офицеров(${m \choose k}$).

Во втором случае мы сначала выбираем k офицеров(${n \choose k}$), а из оставшихся n-k выбираем m-k солдат(${n-k \choose m-k}$).

\subsection{•}

\section{Приведите комбинаторное доказательство равенства $\sum\limits_{0 \geq k \geq \frac{n+1}{2}}{n -k + 1 \choose k} = F_{n+2}$}

Посчитаем количество количество последовательностей из нулей и единиц длины n, таких, что две соседних цифры не еденицы:

В первом случае попробуем вывести рекурентное соотношение($N_k$ - искомые длины), рассмотрим последовательность длины k, в которой первые k-1 элемента можно рассматривать, как какую-то из искомых последовательностей длины k-1, которых $N_{k-1}$, если на последнем месте стоит 1, то на k-1 месте 1 стоять не может, то есть там 0. Тогда в качестве первых k-2 символов возмём любую из искомых последовательностей длины k-2, а их количество $N_{k-2}$, тогда $N_{k} = N_{k-2} + N_{k-1}$. Полученная формула -- формула чисел фибоначи. Посчитаем первые два элемента нашей последовательности $N_{0} = 1$ , a $N_{1} = 2$, по сравнению с числами фибоначи смещение на 2, значит искомое значение -- $F_{n+2}$ .

Во втором случае число единиц в последовательности k, тогда, учитывая, что двух соседних единиц не может быть, $0 \geq k \geq \frac{n+1}{2}$ . Для определенного k: для всех единиц, кроме последней, следующее число равно нулю. Вычеркнем эти нули в количестве $k-1$. Останется $n-k+1$ цифр, среди которых k единиц. Таких последовательностей ${n -k + 1 \choose k}$, проссумируем для всех k $\sum\limits_{0 \geq k \geq \frac{n+1}{2}}{n -k + 1 \choose k} = F_{n+2}$
