\documentclass[a4paper,11pt]{article}

\usepackage[T2A]{fontenc}
\usepackage[utf8]{inputenc}
\usepackage[english,russian]{babel}

\usepackage{amsmath,amsfonts,amssymb,amsthm}

\author{Ахметжанов Руслан}
\title{Задание на 9 неделю}
\date{\today}

\begin{document}

\maketitle 

\section{В корзине лежат 60 шаров различных цветов. Если взять любые 10, то обязательно среди них найдется три одноцветных. Обязательно ли среди всех 60 шаров найдется: 1) 15 одноцветных; 2) 16 одноцветных?}
\subsection{Для обоих вариантов}
условие 1 -- <<Если взять любые 10, то обязательно среди них найдется три одноцветных>>


Так как в вопросе присуствует <<обязательно>>, то нужно рассмотреть минимальное возможное количество шаров одного цвета(плохая ситуация). Понятно, что чем больше цветов, тем меньше шаров каждого цвета. 


Так как самая плохая ситуация - все цвета различны, то самая плохая ситуация при невозможности выполнения предыдущего условия -- максимальное число разных цветов, то есть список цветов входит несколько раз. А для выполнения условия 1 нужно чтобы список входил чуть более чем два раза,  то искомое количество цветов - максимальное целое решение неравенства
\footnote{возможно замечание, что шаров какого-то цвета может быть один, соответственно список второго вхождения на один меньше, но это не повлияет при наших числах, так как $ \frac{60-x}{} $, где $x \geq n$, a n - количество цветов, около пяти при увеличении x будет оставлять слишком много места для шаров не единичного экземпляра, а при маленьких x оно повлияет лишь на один цвет и не изменит картину в целом.}
 $ 2*x < 10 $, то есть максимальное возможное количество цветов -- 4, и соответственно среднее количество шаров одного цвета -- $\frac{60}{4} = 15$,  это значение и будет минимальным, так как, если количество шаров какого-то цвета уменьшить, то количество шаров какого-то другого цвета увеличится.


\subsection{15 одноцветных?}
да

\subsection{16 одноцветных?}
нет

\section{В группе студентов есть один, который знает C++, Java, Python,
Haskell. Каждые три из этих языков знают два студента. Каждые два —
6 студентов. Каждый из этих языков знают по 15 студентов. Каково
наименьшее количество студентов в такой группе?}
Рассмотрим один возможный вариант числа студентов: так как все 4 языка знает 1 студент, то при учитывании числа людей, знающих по 3 языка будем считать, что таких людей по одному, так как общего для всех них мы уже учли, аналошично для людей, знающих по два языка их по $(6-2)$, и для людей, знающих всего 1 язык -- $(15-9)$. То есть всего $1 + 4*(2-1) + 6*(6-2) + 4*(15 - 6) = 65 $ студентов.\\ Может ли быть студентов меньше хотя бы на 1? В таком случае будет общий студент у какого-то из множеств((а)только по 1, (б)только по 3, (в)только по 2) Но если так окажется,  то нарушится одно из условий:
при (а) будет значить, что увеличится число знающих по 2, при (б) по 3, при (в) по 3. А значит, что полученное нами значение Наименьшее
\textbf{Ответ: 65.}


\section{Найдите количество функций алгебры логики от n перемен-
ных, существенно зависящих от всех своих переменных.}

Всего функций от n переменных -- $2^{2^n}$, пусть $L_k$. Вычтем количество функций с K-той фиктивной переменной то есть $(n-1)$ перменных-- $2^{2^{n-1}} \times n$ . Прибавим число попарных пересечений -- $2^{2^{n-2}}$ , их ${n \choose 2}$, и так далее по формуле включений-исключений, то есть всего: \[ \sum\limits_{k=0}^{n}{n \choose k} 2^{2^{n-k}} \]




\section{Сколько существует шестизначных чисел, в записи которых
есть единица и двойка?}

Вместо того, чтобы посчитать, что спрашивают можно посчитать количество чисел без единиц и двоек и вычесть полученное значение из количества всех шестизначных чисел. А количество чисел без единиц и двоек есть объеденение множеств чисел без единиц и без двоек, мощность которого равна разности суммы мощностей чисел без единиц и чисел без двоек с мощностью чисел и без единиц и без двоек:\\* $ (9-1)(10-1)^{6-1} + (9-1)(10-1)^{6-1} - (9-2)(10-2)^{6-1} = 2*8*9^5-7*8^5$. А всего шестизначнхчисел $9*10^5$. Тогда конечный ответ: $ 9*10^5 - 2*8*9^5+7*8^5 = \textbf{184592}$

\section{•}

\section{Сколькими способами можно закрасить часть клеток таблицы $3 \times 4$ так,
чтобы незакрашенные клетки содержали или верхний ряд, или нижний ряд, или
две соседние вертикали?}
Чтобы

 незакрашенные клетки содержали или верхний ряд, или нижний ряд -- $2^{4*(3-1)} \times 2$ 
 
 незакрашенные клетки содержали две соседние вертикали -- $ 3 \times 2^{(4-2)*3} $
Сложим количества для каждого из <<или>>, вычтем пересечения. В силу условий для таблицы у пересечений нет собственных пересечений.
По формуле включений-исключений: 

$X = 2^{9} + 3*2^6 - (2^4*3*2 +2^4+2*2^4+1) = \textbf{559}$

\begin{tabular}{ | l | l | l | l | }
\hline
0 & ? & ? & 0\\ \hline
0 & ? & ? & 0\\ \hline
0 & 0 & 0 & 0\\ \hline
\end{tabular} пример объединения <<или>>









\end{document}