\documentclass[a4paper,11pt]{article}
\usepackage[T2A]{fontenc}
\usepackage[utf8]{inputenc}
\usepackage[english,russian]{babel}

\usepackage{amsmath,amsfonts,amssymb,amsthm}

\author{Ахметжанов Руслан}
\title{Финальное задание}
\date{\today}

\begin{document}

\maketitle 

\textbf{Ex.2} Пусть $F_n$ — это n-ое число Фибоначчи, притом F 0 = 0, F 1 = 1. Докажите, что

 $F_1^2 + F_2^2 + ... + F_n^2 = F_n * F_{n+1}$
 
 \begin{center}
 По определению числа Фибоначи $F_{n+1} = F_n + F_{n-1}$, подствим в данное нам равенство:

  $F_1^2 + F_2^2 + ... + F_n^2 = F_n(F_n + F_{n-1})$
  
  $F_1^2 + F_2^2 + ... + F_n^2 =  F_n^2 + F_n * F_{n-1}$
  
  $F_1^2 + F_2^2 + ... + F_{n-1}^2 = F_n * F_{n-1})$
  
Проделаем такую подстановку ещё раз, то есть вместо $F_n$ подставим $F_{n-1} + F_{n-2} $:
  
   $F_1^2 + F_2^2 + ... + F_{n-1}^2 = F_{n-1}(F_{n-1} + F_{n-2})$
   
  $F_1^2 + F_2^2 + ... + F_{n-1^2} =  F_{n-1}^2 + F_{n-1} * F_{n-2})$
  
  $F_1^2 + F_2^2 + ... + F_{n-1}^2 = F_n * F_{n-1})$
 
Проделаем подстановку ещё (n-1) раз и получим:

$F_1^2 = F_2*f_1$
Так как $F_0 = 0$, $F_1=1$, то $F_2 = 1$, подставим полученное равенство:
1 = 1

Значит равенство верно, а значит и исходное равенство верно, что и требовалось доказать.
\end{center}


\textbf{Ex.3} Решите линейное рекуррентное соотношение второго порядка
\begin{equation*}
	 \begin{cases}
	 A_{k+1} = 4A_k-3A_{k-1},\\
	 A_0 = 1, A_1 = 2.
	 \end{cases}
\end{equation*}

\begin{center}
	$ r^{n+1} = 4r^{n} - 3r^{n-1}$
	
	$ r^{n+2} -4r^{n+1} +3r^n = 0 $
	
	$ r^2 - 4r +3=0 $
	
	$ (r-3)(r-1)=0 $
	
	\begin{equation*}
	 \begin{cases}
  	A_0 = c_1 +c_2	\\
	A_1 = c_1r_1 +c_2r_2
	 \end{cases}
\end{equation*}
	\begin{equation*}
	 \begin{cases}
	 c_2 = 1 - c_1\\
	 2 = 3c_1 + 1 - c_1
	 \end{cases}
\end{equation*}
	 \begin{equation*}
	 \begin{cases}
	 c_1 = 0.5\\
	 c_2 = 0.5
	  \end{cases}
\end{equation*}
	 
	 $ A_n = 0.5(3^n + 1) $
	  
\end{center}


\textbf{Ex.5.} Докажите, что для чисел Белла B n верны тождества

\*
A) \[ \sum\limits_{k=0}^{n}{n \choose k}B_k \]

$ B_{n+1} =  \sum\limits_{r+1}^{n+1} S(n+1, r) = \sum\limits_{r=1}^{n+1} \sum\limits_{k=0}^n{n \choose k} S(k, r-1) $

$ \sum\limits_{k = 0}^n{n \choose k} \sum\limits_{r=1}^{n+1} S(k, r-1) = \sum\limits_{k=0}^n {n \choose k}B_k $



\end{document}