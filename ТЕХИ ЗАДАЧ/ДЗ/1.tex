\documentclass[12pt,a5paper,fleqn]{article}
\usepackage[utf8]{inputenc}
\usepackage{amssymb, amsmath, multicol}
\usepackage[russian]{babel}
\usepackage{graphicx}
\usepackage[shortcuts,cyremdash]{extdash}
\usepackage{wrapfig}
\usepackage{floatflt}
\usepackage{lipsum}
\usepackage{concmath}
\usepackage{euler}
\usepackage{libertine}

\oddsidemargin=-15.4mm
\textwidth=127mm
\headheight=-32.4mm
\textheight=277mm
\tolerance=100
\parindent=0pt
\parskip=8pt
\pagestyle{empty}
\usepackage[normalem]{ulem} % uline
\usepackage{mdframed}
\usepackage{amsthm}
\theoremstyle{definition}
\newtheorem{Def}{Def.}
\newtheorem{Ex}{Ex.}


\flushbottom

\begin{document}
\begin{center}
	{\bf \Large Задание на первую неделю. \\ \normalsize Алгебра логики: введение}
\end{center}
\vspace{0.6cm}

%
%\begin{mdframed}
%\scriptsize
%\begin{Def}
%	\uwave{Булева функция}~---~это отображение $f\colon B_2^n \to B_2$.
%\end{Def}
%\begin{Def}
%	\uwave{Фиктивная переменная функции $f$}~---~это переменная $x_i$ такая,
%	что $$f(x_1, x_2, \ldots, x_{i-1}, 1, x_{i+1}, \ldots, x_n)=
%	f(x_1, x_2, \ldots, x_{i-1}, 0, x_{i+1}, \ldots, x_n)$$ для любых наборов
%	переменных $x_1, x_2, \ldots, x_{i-1}, x_{i+1}, \ldots, x_n$.
%\end{Def}
%\end{mdframed}

\begin{Ex}
	$x,y,z$~---~целые числа, для которых истинно высказывание
$$
	\neg(x=y)\wedge ((y<x)\to(2z>x))\wedge((x<y)\to(x>2z))
$$
	Чему равно $x$, если $z=7,y=16$?
\end{Ex}

\begin{Ex}
	Постройте таблицу истинности для функции
$$
	f(x_1, x_2, x_3) = (x_1 \vee x_2) \downarrow (x_2\to x_3)
$$
	
	и укажите, какие у нее фиктивные переменные, а какие существенные.
\end{Ex}

\begin{Ex}
	Докажите, что
$$
	1 \oplus x_1 \oplus x_2 = (x_1\to x_2)\wedge(x_2\to x_1).
$$
\end{Ex}

\begin{Ex}
	Докажите формулу
$$
	\bigvee_{i,j; i\neq j} x_i \oplus x_j =(x_1\vee x_2\vee\ldots\vee x_n)
	\wedge (\neg x_1\vee\neg x_2\vee\ldots\vee \neg x_n)
$$
\end{Ex}

\begin{Ex}
	Постройте таблицу истинности для $f$ и выразите ее через операции
	$\vee, \wedge, \neg$, если
$$
	f = x_1 \oplus x_2 \oplus x_3 \oplus x_1x_2 \oplus x_1x_3
	\oplus x_2x_3 \oplus x_1x_2x_3.
$$
\end{Ex}

\textbf{Бонусная задача.} Найдите асимптотическую оценку количества ,булевых 
функций от $n$ переменных, которые зависят от всех своих аргументов существенно. 
Иначе говоря, надо придумать такие верхнюю и нижнюю оценки на это количество, 
чтобы их отношение стремилось к $1$ при $n\to\infty$.

\newpage

\setcounter{Ex}{0}

\begin{center}
	{\bf \Large Задание на первую неделю. \\ \normalsize Алгебра логики: введение}
\end{center}
\vspace{0.6cm}

%
%\begin{mdframed}
%\scriptsize
%\begin{Def}
%	\uwave{Булева функция}~---~это отображение $f\colon B_2^n \to B_2$.
%\end{Def}
%\begin{Def}
%	\uwave{Фиктивная переменная функции $f$}~---~это переменная $x_i$ такая,
%	что $$f(x_1, x_2, \ldots, x_{i-1}, 1, x_{i+1}, \ldots, x_n)=
%	f(x_1, x_2, \ldots, x_{i-1}, 0, x_{i+1}, \ldots, x_n)$$ для любых наборов
%	переменных $x_1, x_2, \ldots, x_{i-1}, x_{i+1}, \ldots, x_n$.
%\end{Def}
%\end{mdframed}

\begin{Ex}
	$x,y,z$~---~целые числа, для которых истинно высказывание
$$
	\neg(x=y)\wedge ((y<x)\to(2z>x))\wedge((x<y)\to(x>2z))
$$
	Чему равно $x$, если $z=7,y=16$?
\end{Ex}

\begin{Ex}
	Постройте таблицу истинности для функции
$$
	f(x_1, x_2, x_3) = (x_1 \vee x_2) \downarrow (x_2\to x_3)
$$
	
	и укажите, какие у нее фиктивные переменные, а какие существенные.
\end{Ex}

\begin{Ex}
	Докажите, что
$$
	1 \oplus x_1 \oplus x_2 = (x_1\to x_2)\wedge(x_2\to x_1).
$$
\end{Ex}

\begin{Ex}
	Докажите формулу
$$
	\bigvee_{i,j; i\neq j} x_i \oplus x_j =(x_1\vee x_2\vee\ldots\vee x_n)
	\wedge (\neg x_1\vee\neg x_2\vee\ldots\vee \neg x_n)
$$
\end{Ex}

\begin{Ex}
	Постройте таблицу истинности для $f$ и выразите ее через операции
	$\vee, \wedge, \neg$, если
$$
	f = x_1 \oplus x_2 \oplus x_3 \oplus x_1x_2 \oplus x_1x_3
	\oplus x_2x_3 \oplus x_1x_2x_3.
$$
\end{Ex}

\textbf{Бонусная задача.} Найдите асимптотическую оценку количества ,булевых 
функций от $n$ переменных, которые зависят от всех своих аргументов существенно. 
Иначе говоря, надо придумать такие верхнюю и нижнюю оценки на это количество, 
чтобы их отношение стремилось к $1$ при $n\to\infty$.

\end{document}
