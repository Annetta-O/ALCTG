\documentclass[12pt,a5paper,fleqn]{article}
\usepackage[utf8]{inputenc}
\usepackage{amssymb, amsmath, multicol}
\usepackage[russian]{babel}
\usepackage{graphicx}
\usepackage[shortcuts,cyremdash]{extdash}
\usepackage{wrapfig}
\usepackage{floatflt}
\usepackage{lipsum}
\usepackage{concmath}
\usepackage{euler}
\usepackage{libertine}

\oddsidemargin=-15.4mm
\textwidth=127mm
\headheight=-32.4mm
\textheight=277mm
\tolerance=100
\parindent=0pt
\parskip=8pt
\pagestyle{empty}
\usepackage[normalem]{ulem} % uline
\usepackage{mdframed}
\usepackage{amsthm}
\theoremstyle{definition}
\newtheorem{Def}{Def.}
\newtheorem{Ex}{Ex.}


\flushbottom

\begin{document}
\begin{center}
	{\bf \Large Задание на вторую неделю. \\ \normalsize Математическая логики: множества, \\ фундаментальные понятия
	\\и методы рассуждений}
\end{center}
\vspace{0.4cm}

\begin{Ex}
	Опишите формально множества ($U=\mathbb{Z}$): а) Множества, состоящее из чисел $1, 10$ и $100$;
	б) Множества, состоящие из чисел, больших $5$; в) Множества, состоящее из натуральных чисел,
	меньших $5$; г) Множество, которое не содержит элементов.
\end{Ex}

\begin{Ex}
	Докажите, что для любых множеств $A, B, C$ выполняются равенства 
	
	а) $(A\cup B)\backslash(A\cap B) = (A\backslash B)\cup(B\backslash A)$; 
	б) $(A\cup B)\backslash C = (A\backslash C)\cup(B\backslash C).$
\end{Ex}

\begin{Ex}
	Пусть $A_1\subseteq A_2\subseteq A_3\subseteq\ldots\subseteq A_n\subseteq\ldots $
	~---~невозрастающая последовательность множеств. Известно, что $A_1\backslash A_4
	= A_6\backslash A_9$. Докажите, что $A_2\backslash A_7 = A_3 \backslash A_8$.
\end{Ex}

\begin{Ex}
	Докажите, что число $\sqrt{3} + \sqrt{2}$ иррационально.
\end{Ex}

\begin{Ex}
	Докажите, что для любого целого положительного $n$ выполняется
	$$1\cdot 2^1 + 2 \cdot 2^2 + 3 \cdot 2^3 + \ldots + n \cdot 2^n = (n-1)\cdot 2^{n+1}+2.$$
\end{Ex}

\begin{Ex}
	В прямоугольнике $3\times n$ стоят фишки трех цветов, по $n$ штук каждого цвета.
	Докажите, что можно переставить фишки в каждой строке так, чтобы в каждом столбце 
	были фишки всех цветов.
\end{Ex}


\textbf{Бонусная задача.} Докажите, что для любых положительных чисел
$x_1,\ldots,x_k(k>3)$ выполняется неравенство:
$$\frac{x_1}{x_k + x_2} + \frac{x_2}{x_1 + x_3} + \ldots + 
\frac{x_k}{x_{k-1} + x_1}\geqslant 2.$$

\newpage
\setcounter{Ex}{0}

\begin{center}
	{\bf \Large Задание на вторую неделю. \\ \normalsize Математическая логики: множества, \\ фундаментальные понятия
	\\и методы рассуждений}
\end{center}
\vspace{0.4cm}

\begin{Ex}
	Опишите формально множества ($U=\mathbb{Z}$): а) Множества, состоящее из чисел $1, 10$ и $100$;
	б) Множества, состоящие из чисел, больших $5$; в) Множества, состоящее из натуральных чисел,
	меньших $5$; г) Множество, которое не содержит элементов.
\end{Ex}

\begin{Ex}
	Докажите, что для любых множеств $A, B, C$ выполняются равенства 
	
	а) $(A\cup B)\backslash(A\cap B) = (A\backslash B)\cup(B\backslash A)$; 
	б) $(A\cup B)\backslash C = (A\backslash C)\cup(B\backslash C).$
\end{Ex}

\begin{Ex}
	Пусть $A_1\subseteq A_2\subseteq A_3\subseteq\ldots\subseteq A_n\subseteq\ldots $
	~---~невозрастающая последовательность множеств. Известно, что $A_1\backslash A_4
	= A_6\backslash A_9$. Докажите, что $A_2\backslash A_7 = A_3 \backslash A_8$.
\end{Ex}

\begin{Ex}
	Докажите, что число $\sqrt{3} + \sqrt{2}$ иррационально.
\end{Ex}

\begin{Ex}
	Докажите, что для любого целого положительного $n$ выполняется
	$$1\cdot 2^1 + 2 \cdot 2^2 + 3 \cdot 2^3 + \ldots + n \cdot 2^n = (n-1)\cdot 2^{n+1}+2.$$
\end{Ex}

\begin{Ex}
	В прямоугольнике $3\times n$ стоят фишки трех цветов, по $n$ штук каждого цвета.
	Докажите, что можно переставить фишки в каждой строке так, чтобы в каждом столбце 
	были фишки всех цветов.
\end{Ex}


\textbf{Бонусная задача.} Докажите, что для любых положительных чисел
$x_1,\ldots,x_k(k>3)$ выполняется неравенство:
$$\frac{x_1}{x_k + x_2} + \frac{x_2}{x_1 + x_3} + \ldots + 
\frac{x_k}{x_{k-1} + x_1}\geqslant 2.$$


\end{document}
