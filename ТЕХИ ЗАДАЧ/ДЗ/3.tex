\documentclass[12pt,a5paper,fleqn]{article}
\usepackage[utf8]{inputenc}
\usepackage{amssymb, amsmath, multicol}
\usepackage[russian]{babel}
\usepackage{graphicx}
\usepackage[shortcuts,cyremdash]{extdash}
\usepackage{wrapfig}
\usepackage{floatflt}
\usepackage{lipsum}
\usepackage{concmath}
\usepackage{euler}
\usepackage{libertine}

\oddsidemargin=-15.4mm
\textwidth=127mm
\headheight=-32.4mm
\textheight=277mm
\tolerance=100
\parindent=0pt
\parskip=8pt
\pagestyle{empty}
\usepackage[normalem]{ulem} % uline
\usepackage{mdframed}
\usepackage{amsthm}
\theoremstyle{definition}
\newtheorem{Def}{Def.}
\newtheorem{Ex}{Ex.}


\flushbottom

\begin{document}
\begin{center}
	{\bf \Large Задание на третью неделю. \\ \normalsize Булевы функции. Теорема Поста}
\end{center}
\vspace{0.4cm}

\begin{Ex}
	Постройте СДНФ и СКНФ для функции $(xz+\overline{y})\equiv(x\to y)$.
\end{Ex}

\begin{Ex}
	Постройте замыкание базиса $\{\neg,\oplus\}$.
\end{Ex}

\begin{Ex}
	Укажите существенные и несущественные (фиктивные) переменные функции 
	$f(x_1,x_2,x_3)=00111100$ и разложите ее в ДНФ и КНФ.
\end{Ex}

\begin{Ex}
	Докажите или опровергните полноту системы функций $\{+,\to\}$.
\end{Ex}

\begin{Ex}
	Пусть $f(x_1,\ldots,x_n)$~---~несамодвойственная функция. Докажите, что
	константы $0, 1$ вычисляются в базисе $\{\neg, f\}$.
\end{Ex}

\begin{Ex}
	Запишите в виде КНФ функцию от $n$ переменных, принимающую значение
	$0$ лишь на $\vec{0}$ и на $\vec{1}$. Покажите, что эта функция равна
	дизъюнкции всевозможных скобок $(x_i+x_j)$, где $i\neq j$. 
\end{Ex}

\begin{Ex}
	Функцию алгебры логики называют \textit{симметрической,} если она не 
	меняет своего значения при любой перестановке значений переменных местами.
	Покажите, что функция $\overline{xy}\vee\overline{yz}\vee\overline{zx}$~---~симметрическая.
	Найдите число симметрических функций от $n$ переменных.
\end{Ex}

\begin{Ex}
	Докажите, что любая неконстантная симметрическая функция существенно зависит
	от всех своих переменных.
\end{Ex}

\begin{Ex}
	Докажите, что если система $\{f_1,\ldots,f_n\}$ полна, то и система
	двойственных функций $\{f_1^*, \ldots, f_n^*\}$ также полна.
\end{Ex}

\textbf{Бонусная задача.} Докажите, что штрих Шеффера и стрелка Пирса~---~единственные
функции от двух переменных, через которые выражаются все функции алгебры логики.


\newpage
\setcounter{Ex}{0}

\begin{center}
	{\bf \Large Задание на третью неделю. \\ \normalsize Булевы функции. Теорема Поста}
\end{center}
\vspace{0.4cm}

\begin{Ex}
	Постройте СДНФ и СКНФ для функции $(xz+\overline{y})\equiv(x\to y)$.
\end{Ex}

\begin{Ex}
	Постройте замыкание базиса $\{\neg,\oplus\}$.
\end{Ex}

\begin{Ex}
	Укажите существенные и несущественные (фиктивные) переменные функции 
	$f(x_1,x_2,x_3)=00111100$ и разложите ее в ДНФ и КНФ.
\end{Ex}

\begin{Ex}
	Докажите или опровергните полноту системы функций $\{+,\to\}$.
\end{Ex}

\begin{Ex}
	Пусть $f(x_1,\ldots,x_n)$~---~несамодвойственная функция. Докажите, что
	константы $0, 1$ вычисляются в базисе $\{\neg, f\}$.
\end{Ex}

\begin{Ex}
	Запишите в виде КНФ функцию от $n$ переменных, принимающую значение
	$0$ лишь на $\vec{0}$ и на $\vec{1}$. Покажите, что эта функция равна
	дизъюнкции всевозможных скобок $(x_i+x_j)$, где $i\neq j$. 
\end{Ex}

\begin{Ex}
	Функцию алгебры логики называют \textit{симметрической,} если она не 
	меняет своего значения при любой перестановке значений переменных местами.
	Покажите, что функция $\overline{xy}\vee\overline{yz}\vee\overline{zx}$~---~симметрическая.
	Найдите число симметрических функций от $n$ переменных.
\end{Ex}

\begin{Ex}
	Докажите, что любая неконстантная симметрическая функция существенно зависит
	от всех своих переменных.
\end{Ex}

\begin{Ex}
	Докажите, что если система $\{f_1,\ldots,f_n\}$ полна, то и система
	двойственных функций $\{f_1^*, \ldots, f_n^*\}$ также полна.
\end{Ex}

\textbf{Бонусная задача.} Докажите, что штрих Шеффера и стрелка Пирса~---~единственные
функции от двух переменных, через которые выражаются все функции алгебры логики.





\end{document}
